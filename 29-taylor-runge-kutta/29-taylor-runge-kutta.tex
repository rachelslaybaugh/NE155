\documentclass[12pt]{article}
\usepackage[top=1in, bottom=1in, left=1in, right=1in]{geometry}

\usepackage{setspace}
\onehalfspacing

\usepackage{amssymb}
%% The amsthm package provides extended theorem environments
\usepackage{amsthm}
\usepackage{epsfig}
\usepackage{times}
\renewcommand{\ttdefault}{cmtt}
\usepackage{amsmath}
\usepackage{graphicx} % for graphics files

% Draw figures yourself
\usepackage{tikz} 

% The float package HAS to load before hyperref
\usepackage{float} % for psuedocode formatting
\usepackage{xspace}

% from Denovo Methods Manual
%\usepackage{mathrsfs}
%\usepackage[mathcal]{euscript}
%\usepackage{color}
%\usepackage{array}

\usepackage[pdftex]{hyperref}
\usepackage[parfill]{parskip}

% math syntax
\newcommand{\nth}{n\ensuremath{^{\text{th}}} }
\newcommand{\ve}[1]{\ensuremath{\mathbf{#1}}}
\newcommand{\Macro}{\ensuremath{\Sigma}}

%---------------------------------------------------------------------------
\title{NE 155, Class 30, S17 \\
Taylor Series Methods and Runge-Kutta}
\date{April 5, 2017}
\begin{document}
\author{materials from: Kathryn Huff}
\maketitle

\hrulefill

\section*{Introduction}

Recall that the initial value problem takes the form:
\begin{align}
u'(t) &= f(u(t), t), \mbox{ for } t>t_0
\label{ivp}
\intertext{where}
u(t_0) &= u_0
\end{align}
In such a problem, we desire to compute $u(t_1)$, $u(t_2)$, $u(t_n)$ and so on. 

There are many methods for solving initial value problems numerically. 
This lesson will introduce a simple one, Forward Euler, which is derived from a 
Taylor series expansion. We will then use Forward Euler to introduce a two-stage, explicit 
Runge-Kutta method as well, for higher accuracy.  


%------------------------------------------------------------------------------
\section*{Taylor Series Derivation of Forward Euler}

The simplest method for finding $u(t_n)$ is Forward Euler, which approximates $u(t_n)$. Let's call this approximation, $U^n$. In this notation, Forward Euler is based on replacing $u'(t_n)$ with $(U^{n+1} - U^n)/k$, where $k$ is the width of the timestep.
The Forward Euler method arises from a Taylor series expansion of $u(t_{n+1})$ 
about $u(t_n)$:

\begin{align}
u(t_{n+1}) = u(t_n) + ku'(t_n) + \frac{1}{2}k^2u''(t_n) + \cdots
\label{taylor}
\end{align}

With this, the $O(k^2)$ terms can be dropped to give:
\begin{align}
u(t_{n+1}) \approx u(t_n) + ku'(t_n) 
\end{align}

And, based on equation \eqref{ivp} we can replace $u'(t_n)$ with 
$f(u(t_n),t_n)$:

\begin{equation}
u(t_{n+1}) = u(t_n) + kf(u(t_n),t_n) + H.O.T.
\end{equation}

This expression gives a one-step truncation error of order $O(k^2)$ (that is, this step introduces $O(k^2)$ error) and a global truncation error of $O(k)$ (because you take $T/k$ time steps to get to time $T$). More 
accurate schemes can be derived with a Taylor series expansion by retaining 
higher order terms in equation \eqref{taylor}. Since we are only given 
$u'(t_n) = f(u(t_n),t_n)$, however, the computation of such schemes requires 
repeated recursive differentiation of this function, and can get quite messy.


\section*{Runge-Kutta Methods}
LeVeque, Randall J. \textit{Finite Difference Methods for Ordinary and Partial Differential Equations}. Philadelphia, PA: SIAM, 2007. 

Runge-Kutta is a method used in practice to get a higher order approximation 
\emph{without} explicitly calculating higher order derivatives.

Runge-Kutta uses two stages. The first stage is an update using Euler's method, 
approximating $u(t_{n+1/2})$. 

\begin{align}
U^{n+1/2} = U^n + \frac{1}{2}kf(U^n)\\
\end{align}

The second stage evaluates the function, $f$, at the midpoint to estimate the 
slope.

\begin{align}
U^{n+1} = U^n + kf(U^{n+1/2})
\end{align}

These equations can be combined into a single expression:

\begin{align}
U^{n+1} = U^n + kf(U^n + \frac{1}{2}kf(U^n))
\end{align}

This approximation, because it uses two points, like a centered approximation, 
is order $O(k^3)$ one-step accurate (and $O(k^2)$ globally).

A generic r-stage Runge-Kutta method can be expressed as:
\begin{align}
Y_1 &= U^n + k\sum_{j=1}^r a_{1j}f(Y_j, t_n + c_jk)\\
Y_2 &= U^n + k\sum_{j=1}^r a_{2j}f(Y_j, t_n + c_jk)\\
&\vdots\nonumber\\
Y_r &= U^n + k\sum_{j=1}^r a_{rj}f(Y_j, t_n + c_jk)
\end{align}
\begin{align}
U^{n+1} &= U^n + k\sum_{j=1}^r b_{j}f(Y_j, t_n + c_jk)\\
&\sum_{j=1}^r a_{ij} = c_i\:, \qquad i = 1, 2, \dots, r \\
&\sum_{j=1}^r b_{j} \:.
\end{align}
There are different ways to determine the coefficients that give different orders of accuracy. See textbook referenced for more information.


\section*{Application to PRKE}

Each of these can be applied to the PRKE. In particular, let's consider the 
application of a Forward Euler.

To avoid confusion with the multiplication factor, the width of our timestep will be called $\Delta t$.

\begin{equation}
n(t_{n+1}) = n(t) + \Delta t\left[\frac{\rho(t_n)-\beta}{\Lambda}n(t_n) + \displaystyle\sum^{j=J}_{j=1}\lambda_j\zeta_j\right]
\end{equation}



\end{document}
