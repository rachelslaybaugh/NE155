\documentclass[12pt]{article}
\usepackage[top=1in, bottom=1in, left=1in, right=1in]{geometry}

\usepackage{setspace}
\onehalfspacing

\usepackage[hang,flushmargin]{footmisc} 
% 'hang' flushes the footnote marker to the left,  'flushmargin'  flushes the text as well.

\def\baselinestretch{1}
\setlength{\parindent}{0mm} \setlength{\parskip}{0.8em}

\newlength{\up}
\setlength{\up}{-4mm}

\newlength{\hup}
\setlength{\hup}{-2mm}

\usepackage{amssymb}
%% The amsthm package provides extended theorem environments
\usepackage{amsthm}
\usepackage{epsfig}
\usepackage{times}
\renewcommand{\ttdefault}{cmtt}
\usepackage{amsmath}
\usepackage{graphicx} % for graphics files

% Draw figures yourself
\usepackage{tikz} 

% The float package HAS to load before hyperref
\usepackage{float} % for psuedocode formatting
\usepackage{xspace}

% from Denovo Methods Manual
\usepackage{mathrsfs}
\usepackage[mathcal]{euscript}
\usepackage{color}
\usepackage{array}

\usepackage[pdftex]{hyperref}

\newcommand{\nth}{n\ensuremath{^{\text{th}}} }
\newcommand{\ve}[1]{\ensuremath{\mathbf{#1}}}
\newcommand{\Macro}{\ensuremath{\Sigma}}
\newcommand{\vOmega}{\ensuremath{\hat{\Omega}}}

\begin{document}
\begin{center}
{\bf NE 155, Classes 29 and 30, S14 \\
TE + DE \\ April 9, 2014}
\end{center}

\setlength{\unitlength}{1in}
\begin{picture}(6,.1) 
\put(0,0) {\line(1,0){6.25}}         
\end{picture}

%-------------------------------------------------------------
\section{Math with Random numbers}

You can use random numbers to do math in two primary ways: 
\begin{itemize}
\item sample physical distributions to reproduce physics needed to solve  problems
\item conduct numerical integration to solve problems
\end{itemize}

- History of MC

- Example of physical distribution sampling

- Example of numerical integration

- Explanation of how these ideas can be used to do a calculation

- Relate generic problem solving to nuclear problems

- You would choose to do Monte Carlo when 
\begin{itemize}
\item analytical integration is impossible
\item deterministic methods are too slow, require approximations that don't work, can't get the soln, etc.
\end{itemize} 

- The big drawback of MC methods is that they can take a very long time. We will talk about this more later.


%-------------------------------------------------------------
\section{PDFs and CDFs}


\section{Sampling}


\section{Random Numbers}



\end{document}