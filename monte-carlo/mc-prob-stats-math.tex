\documentclass[12pt]{article}
\usepackage[top=1in, bottom=1in, left=1in, right=1in]{geometry}

\usepackage{setspace}
\onehalfspacing

\usepackage{amssymb}
%% The amsthm package provides extended theorem environments
\usepackage{amsthm}
\usepackage{epsfig}
\usepackage{times}
\renewcommand{\ttdefault}{cmtt}
\usepackage{amsmath}
\usepackage{graphicx} % for graphics files
\usepackage{tabu}

% Draw figures yourself
\usepackage{tikz} 

% writing elements
\usepackage{mhchem}

% The float package HAS to load before hyperref
\usepackage{float} % for psuedocode formatting
\usepackage{xspace}

% from Denovo Methods Manual
\usepackage{mathrsfs}
\usepackage[mathcal]{euscript}
\usepackage{color}
\usepackage{array}

\usepackage[pdftex]{hyperref}
\usepackage[parfill]{parskip}

% math syntax
\newcommand{\nth}{n\ensuremath{^{\text{th}}} }
\newcommand{\ve}[1]{\ensuremath{\mathbf{#1}}}
\newcommand{\Macro}{\ensuremath{\Sigma}}
\newcommand{\rvec}{\ensuremath{\vec{r}}}
\newcommand{\vecr}{\ensuremath{\vec{r}}}
\newcommand{\omvec}{\ensuremath{\hat{\Omega}}}
\newcommand{\vOmega}{\ensuremath{\hat{\Omega}}}
\newcommand{\sigs}{\ensuremath{\Sigma_s(\rvec,E'\rightarrow E,\omvec'\rightarrow\omvec)}}
\newcommand{\el}{\ensuremath{\ell}}
\newcommand{\sigso}{\ensuremath{\Sigma_{s,0}}}
\newcommand{\sigsi}{\ensuremath{\Sigma_{s,1}}}
%---------------------------------------------------------------------------
%---------------------------------------------------------------------------
\begin{document}
\begin{center}
{\bf NE 155, S17\\
April 10, 2017
}
\end{center}


%-------------------------------------------------------
\textbf{Probability Distributions}\\
We need to figure out all of these things about how particles are moving around in our system, but how do we do it?\\
We get functional expressions of the probability that various things will happen and try to take enough samples to effectively capture those expressions.
%
\begin{itemize}
\item For a random variable, $x$, the probability that $x$ will have a value between $a$ and $b$ is $P\{a \leq x \leq b\}$.
%
\item The \textit{probability density fuction} expresses the likelihood that $x'$ will take on a value between $x$ and $x+\Delta x$:
\begin{align*}
\lim_{\Delta x \to 0} f(x)\Delta x &= P \{ x \leq x' \leq x + \Delta x \}\\
\int_a^b f(x) dx &= P\{a \leq x \leq b\}
\end{align*}
%
\item We often normalize this PDF to integrate to one, using one of
\begin{equation}
\int_{-\infty}^{\infty} f(x) dx = 1 \quad \text{or} \quad
\int_{x^-}^{x^+} f(x) dx = 1 \nonumber
\end{equation}
%
\item To get the probability that our random variable $x'$ is less than or equal to some value $x$, we use a \textit{cumulative distribution function}:
\begin{align*}
F(x) &= P\{x' \leq x\} \\
F(x) &= \int_{-\infty}^{x} f(x') dx' \\
\lim_{x \to \infty} F(x) &\equiv F(\infty) = 1 \\
\lim_{x \to -\infty} F(x) &\equiv F(-\infty) = 0 \\
P \{ a \leq x' \leq b \} &= F(b) - F(a)
\end{align*}
\end{itemize}

Various physical phenomena can be represented by probability distributions
\begin{itemize}
  \item Photon emission energy: Each possible energy has a different probability (intensity)
  \item Scattering cross-sections: Each possible scattering angle has a different probability as a function of the energy
  \item Transmission through a medium: Probability of reaching a particular position
depends on the cross-section
\end{itemize}
%
We in one way or another get these PDFs and/or CDFs and use random numbers to select values for use in simulation.


%-------------------------------------------------------
-------------------------------------------------------\\
\textbf{Statistics}\\
The ``true" mean value, $\mu$, of any PDF is the expected value, $E(x)$
\[
\mu = E(x) = \int x f(x) dx
\]
Because we can't usually do this, we use random samples and estimate the true mean from the ``sample" mean, $\bar{x}$
\[
\bar{x} = \frac{1}{N}\sum_{i=1}^N x_i \qquad \lim_{N \to \infty} \bar{x} \rightarrow \mu\:.
\]
The variance of a PDF is the measure of spread in that PDF
\begin{align*}
\sigma^2 &= E[(x - \mu)^2] = \int (x - \mu)^2 f(x) dx \\
&= \int x^2 f(x) dx - 2 \mu \int x f(x) dx + \mu^2 \int f(x) dx\\
&= E(x^2) - \mu^2
\end{align*}
%
However, we don't know the PDF so we use the samples to get the sample variances
\begin{align*}
S_x^2 &= \frac{1}{N-1}\sum_{i=1}^N (x_i - \bar{x})^2 \\
&= \frac{1}{N-1} \biggl[\sum_{i=1}^N x_i^2 - 2 \bar{x}\sum_{i=1}^N x_i + \bar{x}^2 \sum_{i=1}^N 1 \biggr] \\
&\approx \bar{x^2} - \bar{x}^2
\end{align*}

The \underline{Central Limit Theorem} states that
For $N$ \textit{independent} random variables, $x_i$, sampled from \textit{identical distributions}, their mean follows a Normal (Gaussian) distribution.\\
(Note: this is the \textit{IID} requirement for MC)\\
We can use this information to define \textit{confidence intervals}
\begin{align*}
\bar{x} - S_{\bar{x}} &< E(x) < \bar{x} + S_{\bar{x}} \quad \text{about 68\% of the time}\\
\bar{x} - 2S_{\bar{x}} &< E(x) < \bar{x} + 2S_{\bar{x}} \quad \text{about 95\% of the time}
\end{align*}

The \textbf{standard deviation} of the mean is a measure of the error in the result
\begin{align*}
S_{\bar{x}}^2 &= E[(\bar{x} - \mu)^2] = E\biggl[ \biggl(\frac{1}{N}\sum_{i=1}^N x_i - \mu\biggr)^2\biggr] = E\biggl[ \biggl(\frac{1}{N^2}\sum_{i=1}^N (x_i - \mu)\biggr)^2\biggr]\\
%
&=\frac{1}{N^2} E\biggl[ \sum_{i=1}^N (x_i - \mu) \sum_{j=1}^N (x_j - \mu)\biggr] = \frac{1}{N^2} E\biggl[ \sum_{i=1}^N \sum_{j=1}^N (x_i - \mu)  (x_j - \mu)\biggr]\\
%
&= \frac{1}{N^2} \sum_{i=1}^N \sum_{j=1}^N E\bigl[  (x_i - \mu)  (x_j - \mu)\bigr] = \frac{1}{N^2} \sum_{i=1}^N \sum_{j=1}^N S^2_x \delta_{ij} = \frac{1}{N^2} \sum_{i=1}^N S_x^2 \\
%
&= \frac{N S_x^2}{N^2} = \boxed{\frac{S_x^2}{N}}
\end{align*}
The error in the results decreases with the square of increasing the number of histories.
%
\begin{figure}[h!]
\begin{center}
  \includegraphics[height=1 in,clip]{../../NE250/figs/gaussian}
\end{center}
  %\caption{Monte Carlo neutral particle transport algorithm}
  \label{fig:gaussian}
\end{figure}

\textbf{Relative Error} is 
\[
R = \frac{S_{\bar{x}}}{\bar{x}} = \sqrt{\frac{\sum_{i=1}^N x_i^2}{\bigl(\sum_{i=1}^N x_i\bigr)^2} - \frac{1}{N}} 
\]
If $x_i$ are equal and non-zero, R=0.\\
Thus, we can reduce the error by reducing the spread in $x_i$.

\textbf{Accuracy vs.\ Precision}\\
The distinction between accuracy and precision can be seen in Fig.~\ref{fig:accuracy}.\\
%
\begin{figure}[h!]
\begin{center}
  %\includegraphics[height=1 in,clip]{../figs/systematic-error}
  \includegraphics[height=2 in,clip]{../../NE250/figs/accuracy-and-precision}
\end{center}
  \caption{"Accuracy and precision" by Pekaje at English Wikipedia - Transferred from en.wikipedia to Commons.. Licensed under GFDL via Commons - https://commons.wikimedia.org/wiki/File:Accuracy\_and\_precision.svg\#/media/
  File:Accuracy\_and\_precision.svg}
  \label{fig:accuracy}
\end{figure}
%
\textit{Accuracy} is the degree of closeness of measurements of a quantity to that quantity's true value.\\
The \textit{precision} of a measurement system, related to reproducibility and repeatability, is the degree to which repeated measurements under unchanged conditions show the same results.
% https://en.wikipedia.org/wiki/Accuracy_and_precision

Accuracy can be affected by systematic errors in simulation: physical and mathematical models; errors in geometry or source model; incorrect code use by user.\\
Usually unknown.

Conversely, precision can usually be improved: run more histories; use variance reduction; adjust your measurement (fewer scoring bins).

\vspace*{1 em}

\end{document}
