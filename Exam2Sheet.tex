%--------------------------------------------------------------------
% NE 155 (intro to numerical simulation of radiation transport)
% Spring 2017

% formatting
\documentclass[12pt]{article}
\usepackage[top=1in, bottom=1in, left=1in, right=1in]{geometry}

\usepackage{setspace}
\onehalfspacing

\setlength{\parindent}{0mm} \setlength{\parskip}{1em}


% packages
\usepackage{amssymb}
%% The amsthm package provides extended theorem environments
\usepackage{amsthm}
\usepackage{epsfig}
\usepackage{times}
\renewcommand{\ttdefault}{cmtt}
\usepackage{amsmath}
\usepackage{graphicx} % for graphics files

% Draw figures yourself
\usepackage{tikz} 

% The float package HAS to load before hyperref
\usepackage{float} % for psuedocode formatting
\usepackage{xspace}

% from Denovo methods manual
\usepackage{mathrsfs}
\usepackage[mathcal]{euscript}
\usepackage{color}
\usepackage{array}

\usepackage[pdftex]{hyperref}

\newcommand{\nth}{n\ensuremath{^{\text{th}}} }
\newcommand{\ve}[1]{\ensuremath{\mathbf{#1}}}
\newcommand{\macro}{\ensuremath{\Sigma}}
\newcommand{\vOmega}{\ensuremath{\hat{\Omega}}}

\newcommand{\Macro}{\ensuremath{\Sigma}}


\newcommand{\cc}[1]{\ensuremath{\overline{#1}}}
\newcommand{\ccm}[1]{\ensuremath{\overline{\mathbf{#1}}}}


%--------------------------------------------------------------------
%--------------------------------------------------------------------
\begin{document}
% Iterative methods
The \textbf{fixed-point} iterative process and associated error are (where $\rho(\ve{P})$ is the spectral radius of $\ve{P}$.):
\[\vec{x}^{(0)} = \text{ arbitrary}\:; \qquad
\vec{x}^{(k+1)} = \ve{P}\vec{x}^{(k)} + \tilde{\vec{b}} \]
\[\vec{e}^{(k+1)} = \ve{P}\vec{e}^{(k)}\:; 
 \qquad ||\vec{e}^{(k+1)}|| \leq ||\ve{P}^{k+1}||\: ||\vec{e}^{(0)}||\:;
 \qquad || \ve{P}^k || \approx \rho^k (\ve{P})\]
%
To reduce error by a factor of $\epsilon$, it takes $k \approx \frac{\log(\epsilon)}{\log(\rho(\ve{P}))}$ iterations. 
%
\begin{align*}
\text{Richardson} \qquad &\vec{x}^{(k+1)} = (\ve{I} - \omega^{(k)}\ve{A})\vec{x}^{(k)} + \omega^{(k)}\vec{b}\:;
 \qquad x^{(k+1)}_i =  x^{(k)}_i - \omega^{(k)} \sum_{j=1}^{n} a_{ij}x_j^{(k)} + \omega^{(k)} b_i \\
%
\text{Jacobi} \qquad & \vec{x}^{(k+1)} = \ve{D}^{-1}(\ve{D} - \ve{A})\vec{x}^{(k)} + \ve{D}^{-1}\vec{b}\:;
  \qquad x^{(k+1)}_i = \frac{1}{a_{ii}}(b_i - \sum_{j=1}^{i-1} a_{ij} x_j^{(k)} - \sum_{j=i+1}^{n} a_{ij} x_j^{(k)})\\
%
\text{GS} \qquad &  x^{(k+1)}_i = \frac{1}{a_{ii}}(b_i - \sum_{j=1}^{i-1} a_{ij} x_j^{(k+1)} - \sum_{j=i+1}^{n} a_{ij} x_j^{(k)})\\
%
\text{SOR} \qquad & x^{(k+1)}_i = (1-\omega)x_i^{(k)} + \frac{\omega}{a_{ii}}(b_i - \sum_{j=1}^{i-1} a_{ij} x_j^{(k+1)} - \sum_{j=i+1}^{n} a_{ij} x_j^{(k)}) 
\end{align*}
%
\textbf{condition number} of a matrix $\mathbf{A}$ is defined as $\kappa(\mathbf{A}) = ||\mathbf{A}|| \text{ }||\mathbf{A}^{-1}||$.
%
If the 2-norm is used, then $||\mathbf{A}||_{2} = \sigma_{1}$, $||\mathbf{A}^{-1}||_{2} = 1 / \sigma_{m}$, and $\kappa_{2}(\mathbf{A}) = \sigma_{1} / \sigma_{m}$; $\sigma_{m}$ is the $m$th singular value of $\ve{A}$. %If $\mathbf{A}$ is singular, its condition number is infinity. 

Let $\ve{G}$ be a non-singular \textbf{preconditioner}, then $\ve{A}\vec{x}=\vec{b}$ can be transformed as $\ve{G}^{-1}\ve{A}x = \ve{G}^{-1}b$.


\end{document}