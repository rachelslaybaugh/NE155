%--------------------------------------------------------------------
% NE 155 (intro to numerical simulation of radiation transport)
% Spring 2017

% formatting
\documentclass[12pt]{article}
\usepackage[top=1in, bottom=1in, left=1in, right=1in]{geometry}

\usepackage{setspace}
\onehalfspacing

\setlength{\parindent}{0mm} \setlength{\parskip}{1em}


% packages
\usepackage{amssymb}
%% The amsthm package provides extended theorem environments
\usepackage{amsthm}
\usepackage{epsfig}
\usepackage{times}
\renewcommand{\ttdefault}{cmtt}
\usepackage{amsmath}
\usepackage{graphicx} % for graphics files

% Draw figures yourself
\usepackage{tikz} 

% The float package HAS to load before hyperref
\usepackage{float} % for psuedocode formatting
\usepackage{xspace}

% from Denovo methods manual
\usepackage{mathrsfs}
\usepackage[mathcal]{euscript}
\usepackage{color}
\usepackage{array}

\usepackage[pdftex]{hyperref}

\newcommand{\nth}{n\ensuremath{^{\text{th}}} }
\newcommand{\ve}[1]{\ensuremath{\mathbf{#1}}}
\newcommand{\macro}{\ensuremath{\Sigma}}
\newcommand{\vOmega}{\ensuremath{\hat{\Omega}}}

\newcommand{\cc}[1]{\ensuremath{\overline{#1}}}
\newcommand{\ccm}[1]{\ensuremath{\overline{\mathbf{#1}}}}


%--------------------------------------------------------------------
%--------------------------------------------------------------------
\begin{document}
\begin{center}
{\bf NE 155, midterm 1 review S17 \\
March 3, 2017}
\end{center}

\setlength{\unitlength}{1in}
\begin{picture}(6,.1) 
\put(0,0) {\line(1,0){6.25}}         
\end{picture}

%--------------------------------------------------------------------
Here are the topics we've covered and that are fair game for the exam.

The exam will be 50 minutes long and closed book. \\
You may use a calculator.\\
I will provide the notes sheet I'm handing out (and on the course site) with the exam for your use.
%You may have an 8.5" x 11" page with writing on \textbf{ONE} side. You must turn it in with the exam.

I encourage you to think about what can reasonably be asked about in 50 minutes if there are a few questions. 
Also think about what can be asked on an exam at all vs.\ what really requires a computer. 

A goal of mine is for you to understand underlying principles and the meaning behind things. \\
If you understand the meaning most other things will come out of it.
\begin{itemize}
\item Transport and diffusion equation
  \begin{itemize}
  \item meaning of terms
  \item assumptions in derivation
  \item areas of applicability and validity
  \item boundary and interface conditions
  \end{itemize}

\item Interpolation 
  \begin{itemize}
  \item what it is and what it's for  
  \item polynomial (Lagrange based): formula and error calculation 
  \item what we think about when evaluating interpolation quality
  \item piecewise polynomials
  \end{itemize}

\item Approximation using least squares
  \begin{itemize}
  \item what it is and what it's for  
  \item the normal equations
  \end{itemize}

\item Differentiation: Forming expressions for derivatives and their error terms using Taylor's theorem; orders of accuracy as a function of mesh size ($O(h^{x})$)

\item Integration
  \begin{itemize}
  \item Lagrange form of Newton-Cotes
  \item composite Newton-Cotes
  \item both how you derive these rules and compute the errors
  \item quality of integration
  \item closed vs.\ open NC
  \end{itemize}

\item Vectors and properties
  \begin{itemize}
  \item vector norms
  \item measuring error and determining convergence
  \end{itemize}

\item Matrices and properties
  \begin{itemize}
  \item how to compute a determinant; properties of determinants
  \item matrix norms
  \item eigenvalues, eigenvectors, and spectral radius
  \end{itemize}

\item Direct solutions of linear systems ($\ve{A}\vec{x} = \vec{b}$)
  \begin{itemize}
  \item diagonal, lower-tri and upper-tri systems
  \item LU decomposition
  \item tridiagonal systems
  \end{itemize}
\end{itemize}


\end{document}