%%%%%%%%%%%%%%%%%%%%%%%%%%%%%%%%%%%%%%%%%%%%%%%%%%%%%%%%%%%%
%%  Class 1, NE 155
%%

\documentclass[xcolor=x11names,compress]{beamer}

\definecolor{CoolBlack}{rgb}{0.0, 0.18, 0.39}
%% General document %%%%%%%%%%%%%%%%%%%%%%%%%%%%%%%%%%
\usepackage{graphicx}
\usepackage{tikz}
\usetikzlibrary{decorations.fractals}
%%%%%%%%%%%%%%%%%%%%%%%%%%%%%%%%%%%%%%%%%%%%%%%%%%%%%%

%% Beamer Layout %%%%%%%%%%%%%%%%%%%%%%%%%%%%%%%%%%
\useoutertheme[subsection=false,shadow]{miniframes}
\useinnertheme{default}
\usefonttheme{serif}
\usepackage{palatino}

% addition of color
\usepackage{xcolor}
\definecolor{dgreen}{rgb}{0.,0.6,0.}
\definecolor{RawSienna}{cmyk}{0,0.72,1,0.45}

\setbeamerfont{title like}{shape=\scshape}
\setbeamerfont{frametitle}{shape=\scshape}

\setbeamercolor*{lower separation line head}{bg=CoolBlack} 
\setbeamercolor*{normal text}{fg=black,bg=white} 
\setbeamercolor*{alerted text}{fg=red} 
\setbeamercolor*{example text}{fg=black} 
\setbeamercolor*{structure}{fg=black} 
 
\setbeamercolor*{palette tertiary}{fg=black,bg=black!10} 
\setbeamercolor*{palette quaternary}{fg=black,bg=black!10} 

\renewcommand{\(}{\begin{columns}}
\renewcommand{\)}{\end{columns}}
\newcommand{\<}[1]{\begin{column}{#1}}
\renewcommand{\>}{\end{column}}
%%%%%%%%%%%%%%%%%%%%%%%%%%%%%%%%%%%%%%%%%%%%%%%%%%

\begin{document}

%%%%%%%%%%%%%%%%%%%%%%%%%%%%%%%%%%%%%%%%%%%%%%%%%%%%%%
%%%%%%%%%%%%%%%%%%%%%%%%%%%%%%%%%%%%%%%%%%%%%%%%%%%%%%
\begin{frame}
\title{NE 155\\Introduction to Numerical Simulations in Radiation Transport}
\subtitle{Lecture 1: Introduction}
\author{
        Rachel Slaybaugh\\
        \vspace*{1em}
        {\it University of California, Berkeley\\
         Department of Nuclear Engineering}\\
}
\date{January 22, 2014}
\titlepage
\end{frame}

%%%%%%%%%%%%%%%%%%%%%%%%%%%%%%%%%%%%%%%%%%%%%%%%%%%%%%
%%%%%%%%%%%%%%%%%%%%%%%%%%%%%%%%%%%%%%%%%%%%%%%%%%%%%%
\begin{frame}{Office Hours}
\begin{itemize}
\item MWF 11 am to noon
\item Asst. Prof. Slaybaugh, 4173 Etcheverry Hall\\
      Email: slaybaugh@berkeley.edu \\
      Phone: 570-850-3385
\item Office Hours: MF, 2 - 3 pm
\item GSI: 
\item Prerequisites: Math 53 and 54
\item Prerequisite knowledge and skills: 
\begin{itemize}
\item Solve linear, first, and second order differential equations
\item Linear algebra, vector calculus
\item Computer language knowledge (C, C++, FORTRAN), Python or MATLAB	
\end{itemize}
\end{itemize}
\end{frame}

%%%%%%%%%%%%%%%%%%%%%%%%%%%%%%%%%%%%%%%%%%%%%%%%%%%%%%
%%%%%%%%%%%%%%%%%%%%%%%%%%%%%%%%%%%%%%%%%%%%%%%%%%%%%%
\begin{frame}{References}
Textbooks and References:
\begin{itemize}
\item Course notes + handouts
\item E. E. Lewis, W. E. Miller Jr., ``Computational Methods of Neutron Transport," J. Wiley \& Sons, 1984.
\item Ebook: \emph{Numerical Computing with MATLAB} by Cleve Moler (http://www.mathworks.com/moler/index\_ncm.html) OR
\item Choose a Python Ebook that fits your needs:
http://www.leettips.org/2013/02/top-10-free-python-pdf-ebooks-download.html
\end{itemize}
\end{frame}

%%%%%%%%%%%%%%%%%%%%%%%%%%%%%%%%%%%%%%%%%%%%%%%%%%%%%%
%%%%%%%%%%%%%%%%%%%%%%%%%%%%%%%%%%%%%%%%%%%%%%%%%%%%%%
\begin{frame}{Grades and Lab}
Grading
\begin{itemize}
\item \begin{tabular}{ll}
Homework & 40\% \\
Midterms (2) & 15\% + 15\% = 30\% \\
Final Project & 30\% 
\end{tabular}
\item Late submissions: -20\% for each day it is late
\end{itemize}
Class computer lab accounts
\begin{itemize}
\item All students will get class computer lab accounts at Davis Etcheverry Computing Facility
\item DECF (1171 and 1111 Etcheverry): http://www.decf.berkeley.edu/
\item License for MCNP6 (B00004-MNYCP-02) is obtained through RSICC: http://rsicc.ornl.gov/Registration.aspx
\item Apply for the EXE package only
\end{itemize}
\end{frame}

%%%%%%%%%%%%%%%%%%%%%%%%%%%%%%%%%%%%%%%%%%%%%%%%%%%%%%
%%%%%%%%%%%%%%%%%%%%%%%%%%%%%%%%%%%%%%%%%%%%%%%%%%%%%%
\begin{frame}{Course Objectives}
\begin{itemize}
\item Review systems of linear algebraic equations, linear algebra, eigenvalues and eigenvectors of a matrix, spectral radius of a matrix, direct and iterative methods for solving linear systems, numerical differentiation and integration.
\item Introduce the numerical approaches used to solve fixed-source and criticality problems in analysis of neutron transport/diffusion in nuclear systems.
\item Discuss the basic characteristics of deterministic and Monte Carlo approaches to numerical solution of these problems.
\item Illustrate the advantages and disadvantages of various discretization schemes and convergence criteria, and their influence on the accuracy of particular numerical methodology.
\end{itemize}
\end{frame}

%%%%%%%%%%%%%%%%%%%%%%%%%%%%%%%%%%%%%%%%%%%%%%%%%%%%%%
%%%%%%%%%%%%%%%%%%%%%%%%%%%%%%%%%%%%%%%%%%%%%%%%%%%%%%
\begin{frame}{Course Objectives}
\begin{itemize}
\item Introduce the specific features of MCNP, a production level Monte Carlo code for simulation of neutron and photon transport in complex geometries, and illustrate the use of MCNP in various areas of nuclear engineering.
\item Introduction to the SERPENT Monte Carlo code.
\item Develop computational skills that may be required for the upper-division design course (NE 170) and/or graduate-level reactor physics, reactor design or numerical analysis courses.
\item Introduction to parallel computing concepts.
\end{itemize}
\end{frame}

%%%%%%%%%%%%%%%%%%%%%%%%%%%%%%%%%%%%%%%%%%%%%%%%%%%%%%
%%%%%%%%%%%%%%%%%%%%%%%%%%%%%%%%%%%%%%%%%%%%%%%%%%%%%%
\begin{frame}{Relevant Courses}
\begin{itemize}
\item E7 Introduction for Computer Programming for Scientists and Engineers
\item CS4 Introduction to Computing for Engineers
\item CS9A-H various languages for Programmers
\item Math 128A, Numerical Analysis (solution of ordinary differential equations)
\item Math 128B, Numerical Analysis (evaluations of eigenvalues and eigenvectors, solution of simple partial differential equations)
\end{itemize}
\end{frame}

%%%%%%%%%%%%%%%%%%%%%%%%%%%%%%%%%%%%%%%%%%%%%%%%%%%%%%
%%%%%%%%%%%%%%%%%%%%%%%%%%%%%%%%%%%%%%%%%%%%%%%%%%%%%%
\begin{frame}{Websites}
\begin{itemize}
\item Berkeley CSE: http://cse.berkeley.edu/
\item CSE Cloud Computing: http://cloud.citris-uc.org
\item CITRIS CSE Research Theme: http://www.citris-uc.org/research/cse
\item CITRIS Berkeley-CSE: http://www.citris-uc.org/research/cse/calcse	
\item LBNL Berkeley-CSE (CS-Research):  http://www.lbl.gov/cs/cse/index.html
\item CITRIS Program: http://www.citris-uc.org
\item Computing Sciences Directorate, LBNL: http://www.lbl.gov/cs/
\item National Energy Research Scientific Computing Center (NERSC): http://www.nersc.gov	
\item LBNL Computational Research Division (CRD): http://crd.lbl.gov/
\end{itemize}
\end{frame}

%%%%%%%%%%%%%%%%%%%%%%%%%%%%%%%%%%%%%%%%%%%%%%%%%%%%%%
%%%%%%%%%%%%%%%%%%%%%%%%%%%%%%%%%%%%%%%%%%%%%%%%%%%%%%
\begin{frame}{Categories}
\underline{Experiment}: Experimental scientists work by observing how nature behaves\\
\vspace*{0.25 in}
\underline{Theory}: Theoretical scientists use the language of mathematics to explain and predict the behavior of nature\\
\vspace*{0.25 in}
\underline{Computation}: Computational scientists use theoretical and experimental knowledge to create computer based models of aspects of nature
\end{frame}

%%%%%%%%%%%%%%%%%%%%%%%%%%%%%%%%%%%%%%%%%%%%%%%%%%%%%%
%%%%%%%%%%%%%%%%%%%%%%%%%%%%%%%%%%%%%%%%%%%%%%%%%%%%%%
\begin{frame}{Computational Science/Engineering}
\textcolor{dgreen}{Computational Science} seeks to gain understanding principally through the analysis of mathematical models on high performance computers.\\
\vspace*{0.25 in}
The term \textcolor{dgreen}{computational scientists} has been coined to describe scientists, engineers, and mathematicians who apply high performance computer technology in innovative and essential ways to advance the state of knowledge in their respective disciplines.\\
\vspace*{0.25 in}
Thus, we distinguish it from \textcolor{dgreen}{computer science}, which is the study of \emph{computer and computation} and \emph{theory and experiment}, the traditional form of science. 
\end{frame}

%%%%%%%%%%%%%%%%%%%%%%%%%%%%%%%%%%%%%%%%%%%%%%%%%%%%%%
%%%%%%%%%%%%%%%%%%%%%%%%%%%%%%%%%%%%%%%%%%%%%%%%%%%%%%
\begin{frame}{Solving Problems}
\begin{enumerate}
\item Identify the problem
\item Pose the problem in terms of a mathematical model
\item Identify a computational method for solving the model
\item Implement the computational method on computer
\item Asses the answer in the context of the
\begin{itemize}
\item Implementation (computer language and architecture)
\item Method (discrete or continuous)
\item Model (symbolic or numerical)
\end{itemize}
Using
\begin{itemize}
\item Visualization and interpretation
\item Experimental comparisons
\item Analytical comparisons
\item Engineering judgement
\end{itemize}
\end{enumerate}
\end{frame}

%%%%%%%%%%%%%%%%%%%%%%%%%%%%%%%%%%%%%%%%%%%%%%%%%%%%%%
%%%%%%%%%%%%%%%%%%%%%%%%%%%%%%%%%%%%%%%%%%%%%%%%%%%%%%
\begin{frame}{Big Challenges}
\begin{itemize}
\item Science
\begin{itemize}
\item Global climate modeling
\item Astrophysical modeling
\item Biology: genomics; protein folding; drug design
\item Computational Material Sciences and Nanosciences
\end{itemize}
\item Engineering
\begin{itemize}
\item Semiconductor design
\item Earthquake and structural modeling
\item Computation fluid dynamics (airplane design)
\item Combustion (engine design)
\item Analysis and design of nuclear reactors
\end{itemize}
\item Business
\begin{itemize}
\item Financial and economic modeling
\item Transaction processing, web services and search engines
\end{itemize}
\item Defense
\begin{itemize}
\item Nuclear weapons -- test by simulations
\item Cryptography
\end{itemize}
\end{itemize}
\end{frame}

%%%%%%%%%%%%%%%%%%%%%%%%%%%%%%%%%%%%%%%%%%%%%%%%%%%%%%
%%%%%%%%%%%%%%%%%%%%%%%%%%%%%%%%%%%%%%%%%%%%%%%%%%%%%%
\begin{frame}{frame 2}

\end{frame}

%%%%%%%%%%%%%%%%%%%%%%%%%%%%%%%%%%%%%%%%%%%%%%%%%%%%%%
%%%%%%%%%%%%%%%%%%%%%%%%%%%%%%%%%%%%%%%%%%%%%%%%%%%%%%
\begin{frame}{Big Resources?}
\begin{itemize}
\item ``I think there is a world market for maybe five computers."\\
\hspace*{0.5 in}Thomas Watson, chairman of IBM, 1943
\item ``There is no reason for any individual to have a computer in their home."\\
\hspace*{0.5 in}Ken Olson, president and founder of \\
\hspace*{0.5 in}Digital Equipment Corporation, 1977
\item ``640K [of memory] ought to be enough for anybody."\\
\hspace*{0.5 in}Bill Gates, chairman of Microsoft, 1981
\end{itemize}
\end{frame}

%%%%%%%%%%%%%%%%%%%%%%%%%%%%%%%%%%%%%%%%%%%%%%%%%%%%%%
%%%%%%%%%%%%%%%%%%%%%%%%%%%%%%%%%%%%%%%%%%%%%%%%%%%%%%
\begin{frame}{frame 2}

\end{frame}

%%%%%%%%%%%%%%%%%%%%%%%%%%%%%%%%%%%%%%%%%%%%%%%%%%%%%%
%%%%%%%%%%%%%%%%%%%%%%%%%%%%%%%%%%%%%%%%%%%%%%%%%%%%%%
\begin{frame}{frame 2}

\end{frame}

%%%%%%%%%%%%%%%%%%%%%%%%%%%%%%%%%%%%%%%%%%%%%%%%%%%%%%
%%%%%%%%%%%%%%%%%%%%%%%%%%%%%%%%%%%%%%%%%%%%%%%%%%%%%%
\begin{frame}{frame 2}

\end{frame}

%%%%%%%%%%%%%%%%%%%%%%%%%%%%%%%%%%%%%%%%%%%%%%%%%%%%%%
%%%%%%%%%%%%%%%%%%%%%%%%%%%%%%%%%%%%%%%%%%%%%%%%%%%%%%
\begin{frame}{frame 2}

\end{frame}

%%%%%%%%%%%%%%%%%%%%%%%%%%%%%%%%%%%%%%%%%%%%%%%%%%%%%%
%%%%%%%%%%%%%%%%%%%%%%%%%%%%%%%%%%%%%%%%%%%%%%%%%%%%%%
\begin{frame}{frame 2}

\end{frame}

\end{document}