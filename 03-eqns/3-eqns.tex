%\documentclass[12pt, answers]{exam}
\documentclass[12pt]{exam}
\RequirePackage{amssymb, amsfonts, amsmath, latexsym, verbatim, xspace, setspace}
\RequirePackage{tikz, pgflibraryplotmarks}
\usepackage[margin=1in]{geometry}
\usepackage{enumerate, paralist}
\parindent 0ex

\onehalfspacing

\usepackage{amsthm}
\usepackage{epsfig}
\usepackage{times}
\renewcommand{\ttdefault}{cmtt}
\usepackage{graphicx} % for graphics files

% The float package HAS to load before hyperref
\usepackage{float} % for psuedocode formatting

% from Denovo Methods Manual
\usepackage{mathrsfs}
\usepackage[mathcal]{euscript}
\usepackage{color}
\usepackage{array}

\usepackage[pdftex]{hyperref}
\usepackage[parfill]{parskip}

% math syntax
\newcommand{\nth}{n\ensuremath{^{\text{th}}} }
\newcommand{\ve}[1]{\ensuremath{\mathbf{#1}}}
\newcommand{\Macro}{\ensuremath{\Sigma}}

%---------------------------------------------------------------------------
\begin{document}
\begin{center}
{\bf NE 155, Class 3, S21 \\
Types of Equations in the Engineering Fields \\ January 26, 2021}
\end{center}

\setlength{\unitlength}{1in}
\begin{picture}(6,.1) 
\put(0,0) {\line(1,0){6.25}}         
\end{picture}

\noindent \textbf{Introduction}

In science and engineering in general, and nuclear engineering and reactor analysis in specific, we encounter a wide range of mathematical physics equations. In today's lecture we will introduce some of them.
%
\begin{itemize}
\item Ordinary differential equations (ODEs)
\item Partial differential equations (PDEs)
  \begin{itemize}
  \item Elliptic PDEs
  \item Parabolic PDEs
  \item Hyperbolic PDEs
  \end{itemize}
\item Integro-differential equations
\item Integral equations
\end{itemize}

%------------------------------------------------------------------------------
\section{ODEs}
The most general form of an \nth order linear ordinary differential eqn.\ is
%
\ifprintanswers
\begin{equation}
a_{n}(x)y^{(n)}(x) + a_{n-1}(x)y^{(n-1)}(x) + \cdots + a_{2}(x)y^{(2)}(x) + a_{1}(x)y'(x) + a_0(x)y(x) = f(x) \nonumber
\end{equation}
\else
 \\ \\ 
 \hspace*{8em}\textit{Equation here}\\
\fi
%
\noindent where
\begin{itemize}
\item $a_n$ are coefficients
\item $y^{(n)}$ is the \nth derivative of $y$.
\end{itemize}

Boundary conditions:
\begin{enumerate}
\item Initial Value Problem (\textbf{IVP}): if $y$ and its derivatives are given at one end of the domain/interval (e.g.\ time zero if there's time or spatial starting point if there's only space, etc.)
\item Boundary Value Problem (\textbf{BVP}): if $y$ and/or its derivatives are given at \underline{each} end of the interval
\end{enumerate}

\vspace*{1em}
\textbf{Linear 1st order ODE's}

\underline{Reminders}
\begin{itemize}
\item 1st order means that $n=1$. The  coefficients $a_1$ and $a_0$ may depend on $y$ or $y'$.
\item Linear means each coefficient only depends on $x$ (i.e., not on $y$ or derivatives of $y$).
\end{itemize}

%-------------------------------------------------------------
\underline{Linear 1st order ODE Example}:
\ifprintanswers
\begin{equation}
\frac{dy}{dx} + 3y(x) = \sin(x) \qquad x \in [0, 1] \nonumber
\end{equation}
\else
 \\ \\ 
 \hspace*{8em}\textit{Equation here}
\fi
%
%
\begin{itemize}
\item IVP if boundary conditions are $y$(0) = 1; $y$'(0) = 2
\item BVP if boundary conditions are $y$(0)=-1, $y$(1) = 3 
\end{itemize}
%
In this case the general solution is obtained through the use of an integrating factor.

\vspace*{1em}
\noindent \underline{Linear 1st order ODE Example}:

Point Kinetics analysis of a nuclear reactor is an IVP, linear, 1st order ODE.
%
\ifprintanswers
\begin{align}
\frac{dn(t)}{dt} &= \frac{\rho(t) - \beta}{l^*}n(t) + \sum_{i=1}^{N} \lambda_i C_i(t) \nonumber \\
%
\frac{dC_i(t)}{dt} &= \frac{\beta_i}{l^*}n(t) - \lambda_i C_i(t) \qquad i=1,\dots,N \nonumber
\end{align}
\else
 \\ \\ 
 \hspace*{8em}\textit{Equations here}\\ \\
\fi
%
Where (we'll talk  more about what these terms mean later)
%
\begin{itemize}
\item $n$ = \# neutrons / s
\item $\beta$ = fraction of delayed neutrons
\item $\lambda_i$ = effective decay constant of the $i$th precursor
\item $C_i(t)$ = delayed neutron concentration of the $i$th precursor
\item $l^*$ = mean neutron lifetime
\item $\rho = \frac{k-1}{k}$ = reactivity
\end{itemize}
%
BCs: $n(0) = n_0$ and $C_i(0) = C_{i,0}$ for $i=1,\dots,N$.

%-------------------------------------------------------------
\vspace*{1em}
\noindent \underline{Linear 1st order ODE Example}:

The number of atoms in a sample during radioactive decay (assuming decay only here) is described by the Bateman equation, which is a linear, 1st order ODE that is in an IVP:
%
\ifprintanswers
\begin{align}
\frac{dN_1(t)}{dt} &= -\lambda_1 N_1(t) \nonumber \\
\frac{dN_i(t)}{dt} &= -\lambda_i N_i(t) + \lambda_{i-1}N_{i-1}(t) \qquad 1 < i < I \nonumber\\
\frac{dN_I(t)}{dt} &= \lambda_{I-1} N_{I-1}(t) \nonumber \\
\text{BC: }& N_i(t=0) = N_{i,0}\nonumber
\end{align}
\else
 \\ \\ 
 \hspace*{8em}\textit{Equations here}\\ \\
\fi
%
note: isotope $i$ decays into $i+1$. This can be adapted for decay branches, and becomes more complicated if we have neutrons that transmute isotopes. 

%-------------------------------------------------------------
\vspace*{1em}
\underline{2nd order ODE Example}:

\ifprintanswers
\begin{align}
-\frac{d}{dx}p(x) \frac{d}{dx}\phi(x) &+ q(x)\phi(x) = S(x) \nonumber \\
\text{defined for }& \alpha \le x \le \beta \nonumber \\
\text{BC: }& a\frac{d\phi}{dx} + \gamma \phi = \sigma \qquad \text{at } x=\alpha \text{ and } x = \beta \text{ (BVP)}\nonumber
\end{align}
\else
 \hspace*{8em}\textit{Equations here}\\
\fi
%
This has
\begin{itemize}
\item \textbf{Neumann} BCs if $\gamma = 0$ (specifies the values that the \textit{derivative} of a solution is to take on the boundary of the domain)
\item \textbf{Dirichet} BCs: if $a=0$ (specifies the values that a \textit{solution} is to take on the boundary of the domain)
\item \textbf{Mixed} BCs if [$\gamma \ne 0$ and $a = 0$ at $x = \alpha$] and [$a \ne 0$ and $\gamma = 0$ at $x = \beta$] (the solution is required to satisfy a Dirichlet or a Neumann boundary condition in a mutually exclusive way on disjoint parts of the boundary)
\item If $S(x)$ is nonzero at least somewhere over the physical range, a unique solution exists.
\end{itemize}

%-------------------------------------------------------
\vspace*{1 em} \underline{Reminder: eigenpairs}

We can formulate systems of equations as matrix-vector systems that look like \ve{A}$x = \lambda x$. 
\begin{itemize}
\item An eigenvector is a non-zero vector $x$ that, when multiplied by the matrix \ve{A}, yields a constant multiple of $x$. 
\ifprintanswers
\item The constant multiple, $\lambda$, is the eigenvalue corresponding to the eigenvector.
\else
 \item \textit{Text here}
\fi
\item There can be (and usually are) more than one eigenvalue, and more than one eigenvector. 
%\item Several eigenvectors can have the same eigenvalue
\item Sometime the equation can be reformulated as as $y\ve{A} = \alpha y$ (the left eigenvector). In nuclear, we're used to seeing the right eigenvector formulation.
\end{itemize}


%-------------------------------------------------------------
\vspace*{1em}
\underline{2nd order ODE Example}:

The homogeneous equation %[If  $\phi(x)$  is a solution, so is  $c \phi(x)$, where $c$ is an arbitrary (non-zero) constant. Note that in order for this condition to hold, each term in a linear differential equation of the dependent variable $y$ must contain $y$ or any derivative of $y$; a constant term breaks homogeneity]:
%
\begin{align}
-\frac{d}{dx}p(x) \frac{d}{dx}\phi(x) &+ q(x)\phi(x) = \lambda f(x) \phi(x) \nonumber \\
\text{where }p(x) > 0, \:f(x) \geq 0,\: &\lambda = \text{ eigenvalue;}\nonumber \\
\frac{d\phi}{dx} + \gamma_L \phi &= \Macro \qquad \text{at }x=\alpha \qquad \gamma_L \geq 0  \nonumber \\
\frac{d\phi}{dx} + \gamma_R \phi &= \Macro \qquad \text{at }x=\beta \qquad \gamma_R \geq 0\nonumber 
\end{align}
%
is known as the \underline{Sturm-Liouville} eigenvalue problem. (If $\gamma_L = \gamma_R = 0$, the BCs become $\phi(\alpha) = \phi(\beta) = 0$.)

The solution has an infinite number of eigenfunctions, $\phi_i(x)$ with corresponding \textbf{REAL} and \textbf{DISTINCT} eigenvalues, $\lambda_i$. 

\ifprintanswers
If we number them in sequence: $\lambda_0 < \lambda_1 < \lambda_2 < \dots < \lambda_{\infty}$, then $\lambda_0$ is the lowest eigenvalue and $\phi_0(x)$ is the fundamental eigenmode. 

If $q(x) \geq 0$, then all $\lambda_i$ are positive.
\else
 \vspace*{1em} \hspace*{8em} \textit{Text here} 
\fi

%-------------------------------------------------------------
\vspace*{1em}
\underline{2nd order ODE Example}:

1-D, 1-group, time-independent neutron diffusion equation:
%
\begin{align}
-\frac{d}{dx}D(x)\frac{d}{dx}\phi(x) + \Macro_a(x)\phi(x) &= S(x) \qquad \text{Fixed Source} \nonumber \\
-\frac{d}{dx}D(x)\frac{d}{dx}\phi(x) + \Macro_a(x)\phi(x) &= \frac{1}{k} \nu \Macro_f(x) \phi(x)\qquad \text{Fission / Eigenvalue} \nonumber
\end{align}
%
BCs: (BVP) vacuum, $\phi(\pm a) = 0$


\textbf{Aside:} Recall
%
\begin{align}
\text{gradient is } \nabla T &= \vec{i}\frac{\partial T}{\partial x} + \vec{j}\frac{\partial T}{\partial y} + \vec{k}\frac{\partial T}{\partial z} \nonumber \\
%
\text{divergence is } \nabla \cdot \vec{v} &= \frac{\partial v_x}{\partial x} + \frac{\partial v_y}{\partial y} + \frac{\partial v_z}{\partial z} \nonumber
\end{align}

%-------------------------------------------------------------
\section{PDEs}

A partial differential equation is an equation containing an unknown function of two or more variables and its derivatives with respect to those variables. 

If the PDE is linear in $u$ and all derivatives of $u$, then we say that the PDE is linear.
%
\ifprintanswers
\begin{equation}
A\frac{\partial^2 u}{\partial x^2} + B\frac{\partial^2 u}{\partial x \partial  y} + C\frac{\partial^2 u}{\partial y^2} + D\frac{\partial u}{\partial x} + E\frac{\partial u}{\partial y} + Fu(x,y) = G \nonumber
\end{equation}
\else
 \\ \\ \hspace*{8em}\textit{Equation here}\\ \\
\fi
%
This equation is a 2nd order PDE in two variables. It is linear if $A$ through $G$ do not depend on $u$ (they may depend on $x$ and/or $y$). 
%If $A^2 + B^2 + C^2 > 0$ over a region of the $xy$ plane, the PDE is second-order in that region (analogous to the eqn.\ for a conic section: $Ax^2 + Bxy + Cy^2 + \cdots = 0$. 

%-------------------------------------------------------------
%-------------------------------------------------------------
\vspace*{1em}
\underline{Classification of PDEs}:%http://en.wikipedia.org/wiki/Partial_differential_equation#Classification

Just as one classifies conic sections and quadratic forms into parabolic, hyperbolic, and elliptic based on the discriminant $B^2 - 4AC$, the same can be done for a second-order PDE at a given point. 

[To think about classification, think about replacing $\partial x$ by $x$ and  $\partial y$ by $y$ (formally this is done via Fourier transform). This converts the PDE into a polynomial of the same degree.]

\textit{Note:} these classifications only apply to second order PDEs. 

\noindent The reason we care about this in the context of the Transport Equation:
\ifprintanswers
\begin{itemize}
\item In a void, the transport equation is like a \textit{hyperbolic} wave equation. 
\item For highly-scattering regions where $\Macro_{s}$ is close to $\Macro$, the equation becomes \textit{elliptic} for the steady-state case. 
\item If the scattering is forward-peaked then the equation is \textit{parabolic}.
\end{itemize}
\else
 \\ \\  \hspace*{8em} \textit{Text here} 
\fi

Let's look at the classifications:
%-------------------------------------------------------------
\begin{itemize}
\item \textbf{Elliptic} if $B^2 - 4 AC < 0$. %Solutions of elliptic PDEs are as smooth as the coefficients allow within the interior of the region where the equation and solutions are defined. For example, solutions of Laplace's equation are analytic within the domain where they are defined, but solutions may assume boundary values that are not smooth. The motion of a fluid at subsonic speeds can be approximated with elliptic PDEs, and the Euler–Tricomi equation is elliptic where $x < 0$. 

Some famous elliptic PDEs:
\begin{align*}
\nabla^2 u &= 0 \qquad \text{Laplace's eqn.} \nonumber \\
\nabla^2 u &= f(x) \qquad \text{Poisson's eqn.} \nonumber \\
-\frac{\partial}{\partial x}D(x,y)\frac{\partial}{\partial x}\phi(x,y) &- \frac{\partial}{\partial y}D(x,y)\frac{\partial}{\partial y}\phi(x,y) + \bigl(\Macro_a(x,y) - \frac{1}{k} \nu \Macro_f(x,y)\bigr) \phi(x,y) = 0\nonumber
\end{align*}
For each of these there's no $B$ term, so $-4AC < 0$ (in the diffusion equation case since $D(x,y)$ is positive).

One property of constant coefficient elliptic equations is that their solutions can be studied using the Fourier transform. %http://mathworld.wolfram.com/EllipticPartialDifferentialEquation.html

 %------------------------------------------------------------- 
\item \textbf{Parabolic} if $B^2 - 4 AC = 0$, e.g.
\ifprintanswers
\begin{align}
\frac{\partial u}{\partial t} &= k \frac{\partial^2 u}{\partial x^2} \qquad \text{ 1-D heat eqn.} \nonumber \\
\frac{1}{v}\frac{\partial \phi(x,t)}{\partial t} &= \frac{\partial}{\partial x}D(x,t)\frac{\partial}{\partial x}\phi(x,t) + \bigl(\nu \Macro_f(x,t)\ - \Macro_a(x,t)\bigr) \phi(x,t) + S(x,t)\nonumber
\end{align}
\else
 \\ \\ \hspace*{8em}\textit{Equations here}\\ \\
\fi
There aren't $B$ or $C$ terms, so $-4AC = 0$

Equations that are parabolic at every point can be transformed into a form analogous to the heat equation by a change of independent variables. Solutions smooth out as the transformed time variable increases.
\end{itemize}

A perturbation of the initial (or boundary) data of an \textit{elliptic or parabolic} equation is felt at once by essentially all points in the domain. 

%-------------------------------------------------------------
\begin{itemize}
\item \textbf{Hyperbolic} if $B^2 - 4 AC > 0$, e.g.
\begin{align*}
\frac{\partial^2 u}{\partial t^2} &- c^2\frac{\partial^2 u}{\partial x^2} = 0 \qquad \text{ 1-D wave eqn.} \nonumber
\end{align*}
There's no $B$ term, and the $C$ term is negative so $-4AC > 0$
  \begin{itemize}
  \item if $u$ and its first $t$ derivative are arbitrarily specified with initial data on the initial line $t= 0$ (with sufficient smoothness properties), then there exists a solution for all of $t$.
  \item The solutions of hyperbolic equations are ``wave-like." If a disturbance is made in the initial data of a hyperbolic differential equation, then not every point of space feels the disturbance at once.
  \item Relative to a fixed time coordinate, disturbances have a finite propagation speed. They travel along the characteristics of the equation.
  \end{itemize}
 
%The Euler–Tricomi equation has parabolic type on the line where $x = 0$.
\end{itemize}

%-------------------------------------------------------------
\vspace*{1 em}  \underline{higher order PDE classification}\\
If there are $n$ independent variables $x_1, x_2 , \dots, x_n$, a general linear partial differential equation of second order has the form
%
\begin{equation}
Lu = \sum_{i=1}^n \sum_{j=1}^n a_{i,j} \frac{\partial^2 u}{\partial x_i x_j} + \text{ Lower Order Terms } = 0 \nonumber
\end{equation}
%
The classification depends upon the signature of the eigenvalues of the coefficient matrix $a_{i,j}$.
\begin{enumerate}
\item \underline{Elliptic}: The eigenvalues are all positive or all negative.
\item \underline{Parabolic}: The eigenvalues are all positive or all negative, save one that is zero.
\item \underline{Hyperbolic}: There is only one negative eigenvalue and all the rest are positive, or there is only one positive eigenvalue and all the rest are negative.
\item \underline{Ultrahyperbolic}: There is more than one positive eigenvalue and more than one negative eigenvalue, and there are no zero eigenvalues. There is only limited theory for ultrahyperbolic equations (Courant and Hilbert, 1962).
\end{enumerate}

%%-------------------------------------------------------------
%\vspace*{1em}
%\noindent \underline{Elliptic example}:
%
%\noindent Poisson's equation (stationary heat conduction with a volumetric source):
%%
%\begin{equation}
%\nabla \cdot (k \nabla T) + q''' = 0 \nonumber
%\end{equation}
%% 
%where
%\begin{itemize}
%\item $T$ is temperature
%\item $k$ is thermal conductivity
%\item $q'''$ is volumetric heat generation rate
%\end{itemize}
%%
%There's no $B$ term, so $-4AC < 0$ since $k$ is positive.
%
%%\begin{itemize}
%\item Dirichlet BCs if $f(a,b) =$ constant
%\item Neumann BCs if 
%\begin{equation}
%\vec{n} \cdot \nabla f = n_x \frac{\partial f} {\partial x} + n_y \frac{\partial f} {\partial y}  +n_z \frac{\partial f} {\partial z} \nonumber
%\end{equation}
%\end{itemize}
%
%%-------------------------------------------------------------
%\vspace*{1em}
%\noindent \underline{Parabolic Examples}:
%
%Transient heat conduction in a slab:
%%
%\begin{equation}
%\frac{\partial T(x,t)}{\partial t} = \frac{\partial}{\partial x} \frac{k}{\rho c} \frac{\partial}{\partial x} T(x,t) + q'''(x,t) \nonumber
%\end{equation}
%%
%where $\rho$ = mass density and $c$ = specific heat capacity
%
%\noindent There aren't $B$ or $C$ terms, so $-4AC = 0$
%
%
%\vspace*{2em}
%(2) Time-dependent, 1-group neutron diffusion equation
%%
%\begin{equation}
%\frac{1}{v} \frac{\partial \phi(x,t)}{\partial t} = \frac{\partial}{\partial x} D \frac{\partial}{\partial x} \phi(x,t) + (\nu \Macro_f - \Macro_a) \phi(x,t) + S(x,t) \nonumber
%\end{equation}
%%
%There aren't $B$ or $C$ terms, so $-4AC = 0$
%
%%-------------------------------------------------------------
%\vspace*{1em}
%\noindent \underline{Hyperbolic Example}:
%
%(1) The wave equation
%%
%\begin{equation}
%\frac{\partial^2 f}{\partial t^2} - c \frac{\partial^2 f}{\partial x^2} = 0
%\end{equation}
%%
%where $f(x,t)$ is the displacement and $c$ is the speed of wave propagation.
%
%\noindent There's no $B$ term, and $C$ is negative so $-4AC > 0$
%
%%\noindent \textbf{1st order Hyperbolic equations}
%
%For a system of the form
%%
%\begin{equation}
%\frac{\partial u}{\partial t} + A \frac{\partial u}{\partial x} + B u = F(t,x)
%\end{equation}
%%
%is hyperbolic if the matrix $A$ is diagonalizable with real eigenvalues.
%
%Diagonalizable means there is a nonsingular matrix $P$ such that
%%
% \begin{align}
%    PA^{-1}P &=    &=    \begin{pmatrix}
%      a_{1} & 0 & 0 & \cdots & 0 \\
%      0 & a_{2} & 0 & \cdots & 0 \\
%      0 & 0 & a_{3} & \cdots & 0 \\
%      \vdots & \vdots & \vdots & \ddots   & \vdots \\
%      0 & 0 & 0 & \cdots & a_{d} \\
%    \end{pmatrix} \nonumber  & = \lambda
%    %
%\end{align} 
%http://www.siam.org/books/textbooks/OT88sample.pdf
%
%%\vspace*{2em}
%(2) 1-D, unsteady flow of compressible fluids (in a uniform conduit)
%%
%\begin{equation}
%\frac{\partial \rho}{\partial t} + \gamma \frac{\partial \rho}{\partial x} + \rho \frac{\partial v}{\partial x} = 0
%\end{equation}
%
%where $\rho$ is the fluid density and $v$ is the fluid velocity.

%\vspace*{2em}
%(2) The equation of motion under the influence of gravity 




\end{document}