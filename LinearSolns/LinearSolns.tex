%--------------------------------------------------------------------
% NE 155 (intro to numerical simulation of radiation transport)
% Spring 2014

% formatting
\documentclass[12pt]{article}
\usepackage[top=1in, bottom=1in, left=1in, right=1in]{geometry}

\usepackage{setspace}
\onehalfspacing

\setlength{\parindent}{0mm} \setlength{\parskip}{1em}


% packages
\usepackage{amssymb}
%% The amsthm package provides extended theorem environments
\usepackage{amsthm}
\usepackage{epsfig}
\usepackage{times}
\renewcommand{\ttdefault}{cmtt}
\usepackage{amsmath}
\usepackage{graphicx} % for graphics files

% Draw figures yourself
\usepackage{tikz} 

% The float package HAS to load before hyperref
\usepackage{float} % for psuedocode formatting
\usepackage{xspace}

% from Denovo methods manual
\usepackage{mathrsfs}
\usepackage[mathcal]{euscript}
\usepackage{color}
\usepackage{array}

\usepackage[pdftex]{hyperref}

\newcommand{\nth}{n\ensuremath{^{\text{th}}} }
\newcommand{\ve}[1]{\ensuremath{\mathbf{#1}}}
\newcommand{\macro}{\ensuremath{\Sigma}}
\newcommand{\vOmega}{\ensuremath{\hat{\Omega}}}

\newcommand{\cc}[1]{\ensuremath{\overline{#1}}}
\newcommand{\ccm}[1]{\ensuremath{\overline{\mathbf{#1}}}}


%--------------------------------------------------------------------
%--------------------------------------------------------------------
\begin{document}
\begin{center}
{\bf NE 155, Classes 8 \& 9, S14 \\
Solutions for Linear Systems \\ February 7 and 10, 2014}
\end{center}

\setlength{\unitlength}{1in}
\begin{picture}(6,.1) 
\put(0,0) {\line(1,0){6.25}}         
\end{picture}

\section{Solutions of Linear Systems}

Given $\ve{A}\vec{x} = \vec{b}$, how do you find $\vec{x}$?
%
\begin{enumerate}
\item Direct Methods
  \begin{itemize}
  \item compute (explicitly or implicitly) $\ve{A}^{-1} \rightarrow \vec{x} = \ve{A}^{-1}\vec{b}$
  \item typically requires many operations (FLOPs), large storage (even if $\ve{A}$ is sparse, $\ve{A}^{-1}$ may not be)
  \item seldom parallelizes well
  \item \emph{not} used in practice
  \item incomplete factorizations (which we'll cover) \emph{are} used as pre-conditioners for iterative methods
  \end{itemize}
  
\item Iterative Methods
  \begin{itemize}
  \item produce a sequence of vectors, $\vec{x}_1, \vec{x}_2, \dots$ based on the prescription
  \[\vec{x}^{(k+1)} = F(\vec{x}^{(k)}, \vec{b}) \qquad \text{where } \displaystyle \lim_{k \rightarrow \infty} \vec{x}^{(k)} = \vec{x}\]
  \item can require many FLOPs; storage requirements for $\ve	{A}$ can be an issue (if stored)
  \item may parallelize well
  \item \emph{often} used in practice
  \item used in smoothing methods and as pre-conditioners
  \item Jacobi, Gauss-Seidel, SOR, Multigrid, ...
  \end{itemize}
\end{enumerate}

%\subsubsection{LU Decomposition}
%We can often decompose $\ve{A}$ into an upper and lower triangular matrix
%
%\begin{align}
%\ve{A} &= \ve{L}\ve{U} \qquad \text{and then} \nonumber \\
%%
%\text{det}(\ve{A}) &= \text{det}(\ve{L}) \text{det}(\ve{U}) \qquad \text{and} \nonumber \\
%%
%\ve{A}^{-1} &= \ve{U}^{-1}\ve{L}^{-1} \nonumber
%\end{align}



\end{document}