\documentclass[12pt]{article}
\usepackage[top=0.75in, bottom=0.75in, left=1in, right=1in]{geometry}
\pagestyle{empty}
\usepackage{tabu}
\PassOptionsToPackage{hyphens}{url}
\usepackage{hyperref}
\usepackage{csvsimple}

\renewcommand{\thefootnote}{\fnsymbol{footnote}}
\begin{document}

\begin{center}
{\bf NE 155 - Introduction to Numerical Simulations in Radiation Transport \\ 
T Th 11:00-12:30; Zoom
}
\end{center}

\setlength{\unitlength}{1in}
\begin{picture}(6,.1) 
\put(0,0) {\line(1,0){6.25}}         
\end{picture}

\renewcommand{\arraystretch}{2}

\vskip.25in
\noindent\textbf{Instructor:} Rachel Slaybaugh\\ 
\hspace*{0.95 in}slaybaugh@berkeley.edu,  570-850-3385
\vskip.25in
\noindent\textbf{Office Hours:} 2:00-3:00 PM Mondays and 11:00 AM-noon Fridays.\\
\url{https://berkeley.zoom.us/j/95291780596?pwd=N3lacGI3bFNPUHRwQVhPdFo1dE5tUT09}\\
Meeting ID: 952 9178 0596\\
Passcode: 983248\\
One tap mobile: +16699006833,,95291780596\#,,,,,,0\#,,983248\# US (San Jose)\\
Find your local number: \url{https://berkeley.zoom.us/u/abKMI7vxF7}

%\vskip.25in
%\noindent\textbf{GSI:} TBD
%\vskip.25in
%\noindent\textbf{GSI Office Hours:} TBD

\vskip.25in
\noindent\textbf{Course Location:} Zoom Meeting\\
\url{https://berkeley.zoom.us/j/95302616891?pwd=S0dhNXpnNGxVWjQ2R0doSFlWc1hEdz09}\\
Meeting ID: 953 0261 6891\\
Passcode: 906273\\
One tap mobile: +16692192599,,95302616891\#,,,,,,0\#,,906273\# US (San Jose)\\
Find your local number: \url{https://berkeley.zoom.us/u/adclHJoL9T}


\vskip.25in
\noindent\textbf{Course Description:}
Computational methods used to analyze radiation transport described by various differential, integral, and integro�differential equations. Numerical methods include finite difference, finite volume, discrete ordinates, and Monte Carlo. Examples from neutron and photon transport; numerical solutions of neutron/photon diffusion and transport equations. %Monte Carlo simulations of photon and neutron transport. An overview of optimization techniques for solving the resulting discrete equations on vector and parallel computer systems.

\vskip.25in
\noindent\textbf{bCourses Site:} \url{https://bcourses.berkeley.edu/courses/1501866}\\

\noindent\textbf{Course GitHub page:} \url{https://github.com/rachelslaybaugh/NE155}

\vskip.25in
\noindent\textbf{Prerequisites:} Math 53 and 54 (Eng 7 or basic programming skills strongly recommended)

\vskip.25in
\noindent\textbf{References:}
\begin{itemize}
\item Course notes + handouts: \url{https://github.com/rachelslaybaugh/NE155}
%\item E. E. Lewis, W. E. Miller Jr., ``Computational Methods of Neutron Transport," J. Wiley \& Sons, 1984.
\item Helpful Resources \url{http://tinyurl.com/ne-technical-resources}
\item Free Python Ebooks that might fits your needs: \url{https://pythonbooks.revolunet.com/l}
\item Jupyter: \url{http://jupyter.org/}
\end{itemize}

\clearpage
\noindent\textbf{Computer Information:} 
\begin{itemize}
  \item All students will get class computer lab accounts at Davis Etcheverry Computing Facility (DECF) (1171 and 1111 Etcheverry): \href{http://www.decf.berkeley.edu/}{http://www.decf.berkeley.edu/}
  \item A package with Python, Jupyter, and many useful support libraries (called Anaconda) can be downloaded from \url{https://www.anaconda.com/products/individual}
  \item Software Carpentry has useful lessons: \href{http://software-carpentry.org/lessons.html}{http://software-carpentry.org/lessons.html}
  \item We may use the Serpent Monte Carlo code (\href{http://montecarlo.vtt.fi/}{http://montecarlo.vtt.fi/}) in this course
\end{itemize}

\noindent\textbf{Schedule}: \textit{on bcourses; all dates are subject to change}\\

\noindent\textbf{Grading:} 
\begin{itemize}
\item Worth of activities
  \begin{itemize}
  \item Midterms (2) 30\% + 30\% = 60\%
  \item Final Project 40\% 
  \end{itemize}
\item Grade scale we'll be trying a 5-point grading scale this semester
  \begin{itemize}
  \item 4: A, demonstrates mastery of topics (shows work illustrating the correct solution was achieved with correct steps)
  \item 3: B, demonstrates knowledge about topics (answer may be incorrect, but many steps are correct / pointing in the right direction)
  \item 2: C, demonstrates some understanding of topics (some key steps may be missing or incorrect, but work is headed in a valid direction)
  \item 1: D, demonstrates very limited understanding of topics (some steps are correct, and some correct approaches are included) 
  \item 0: F, does not demonstrate understanding of topics (most steps are incorrect, or the approach taken is not valid for the problem)
  \end{itemize}
\end{itemize}

\noindent There will not be graded homework this semester. I will release homework assignments and then when the homework would normally be due I will release the solutions. I encourage you to treat this like normal homework and then assess yourself to see how you did. I will hold two office hours each week to answer homework questions rather than just one. 

%\vskip.25in
%\noindent\textbf{the Hacker Within:} 
%\begin{itemize}
%\item Tuesdays, 4-5:30 pm, 190 Doe Library (BIDS Space)
%\item Will teach skills useful for this course
%\item Website: \href{http://thehackerwithin.github.io/berkeley/}{http://thehackerwithin.github.io/berkeley/}
%\item GitHub: \href{https://github.com/thehackerwithin/berkeley}{https://github.com/thehackerwithin/berkeley}
%\end{itemize}


\vspace*{.15in}
\noindent \textbf{Course Outline:} 
\begin{enumerate}
\item Introduction
  \begin{enumerate}
  \item Overview of computational science/engineering 
  \item History of computing and parallelization
  \item Types of differential and integral equations in radiation transport
%    \begin{enumerate}
%    \item Integro-�differential form of transport equation 
%    \item Integral form of transport equation 
%    \item Diffusion approximation to transport equation 
%    \item Point �kinetics equation; depletion equation
%    \end{enumerate}
  \item Overview of numerical simulations: deterministic and probabilistic methods
  \end{enumerate}

\item Numerical analysis fundamentals
  \begin{enumerate}
  \item Vector and matrix properties
  \item Eigenvalues and eigenvectors of a matrix; spectral radius of a matrix; convergence of vectors and matrices
  \item Interpolation and polynomial approximation
  \item Numerical differentiation and integration
  \item Direct methods for solving linear systems; Gaussian elimination; pivoting strategies; techniques for special matrices
  \item Iterative methods for solving linear systems: Jacobi, Gauss �Seidel and SOR 
  \end{enumerate}

\item Neutron diffusion equation in 1-D: numerical solution of 2nd order ODEs
  \begin{enumerate}
  \item Derivation of the transport and diffusion equations
  \item Formulation of the finite difference and volume equations for the ``fixed-source" problem
  \item Direct solution by Gaussian elimination
  \item Iterative solutions by Jacobi, Gauss �Seidel, and SOR 
  \item Formulation of the finite difference and volume equations for the eigenvalue (criticality) problem
  \item Power and inverse power iterative methods
  \item Extension to 2-D
  \end{enumerate}

\item Point �kinetics equation: numerical solution of initial value problem
  \begin{enumerate}
  \item Taylor method
  \item Runge�Kutta method
  \item Predictor-corrector methods
  \end{enumerate}

\item Probabilistic numerical simulations: Monte Carlo method
%  \begin{enumerate}
%  \item Continuous and discrete probability distribution; probability density functions; cumulative probability distribution functions
%  \item Random numbers; categories of random sampling
%  \item Complex geometry description and ray tracing
%  \item Analog Monte Carlo; non-analog Monte Carlo; importance sampling;  variance reduction methods; error estimates
%  \item Monte Carlo simulation of neutron and photon transport
%  \item Introduction to the MCNP or Serpent code
%  \end{enumerate} 
    
\item Neutron transport equation in 1-D: numerical solution of integro-differential equations
%  \begin{enumerate}
%  \item Spatial discretization in slab geometry: diamond �difference, step difference, and step characteristic methods
%  \item Angular discretization: discrete �ordinates ($S_N$) method, some $S_N$ Gauss-�Legendre quadrature sets
%  \item Solution of fixed-�source problems with no scattering
%  \item Iterative methods for solving discretized equations
%  \item Source iteration for k-�eigenvalue problems
%  \item Convergence of source iteration method
%  \item Multidimensional discrete ordinates ($S_N$) methods (angular quadrants, ray effects, streaming effects)
%  \end{enumerate}
\end{enumerate}

\vspace*{.15in}
\noindent\textbf{Academic Honesty}:  Berkeley's honor code is

\begin{quote}
As a member of the UC Berkeley community, I act with honesty, integrity, and respect for others.
\end{quote}

\noindent The University provides some basic guidance about academic integrity: \url{http://sa.berkeley.\\edu/conduct/integrity}. Lack of knowledge of the academic honesty policy is not a reasonable explanation for a violation. Questions related to the academic honesty policy should be directed to me.

%\vskip.25in
%\noindent My policy is that you may work together on homework, \textit{but you must specifically cite with whom you worked and what you did together}.

\vskip.25in
\noindent\textbf{Extra Help}:  Do not hesitate to come to my office during office hours or by appointment to discuss a homework problem or any aspect of the course. 

\vskip.25in
\noindent\textbf{Attendance}: Students are encouraged to attend classes regularly. I recognize this is a strange semester and you may be watching the videos asynchronously. %A student who incurs an excessive number of absences may be withdrawn from this class at my discretion.

\vskip.25in
\noindent\textbf{Everyone is welcome}:
Diversity is a defining feature of the University of California and we embrace it as a source of strength. Our differences of race, ethnicity, gender, religion, sexual orientation, gender identity, age, socioeconomic status, abilities, experience and more, enhance our ability to achieve the university?s core missions of public service, teaching, and research. In this class we embrace diversity and welcome students from all backgrounds and want everyone in the room to feel respected and valued.

\vskip.25in
\noindent\textbf{Code of Conduct}:
We'll abide by this code of conduct:\\ \url{https://www.goodenergycollective.org/code-of-conduct}



\vskip.25in
\noindent\textbf{Other Policies and Resources}: 
\begin{itemize}
  \item Student mental health resources:\\ \url{http://www.uhs.berkeley.edu/students/counseling/cps.shtml} 
  \item Sexual assault support on campus: \url{http://survivorsupport.berkeley.edu/}
  \item Accommodation of religious creed: \url{https://sa.berkeley.edu/uga/religion}
  \item Conflicts between extracurricular activities and academic requirements:\\ \url{https://teaching.berkeley.edu/checklist-scheduling-conflicts-academic-requirements}
  \item In case of illness, personal issue, or life event that may impact your performance please let me know as soon as possible so that we can make appropriate arrangements. Things happen and we can work around them; no need to make your life more complicated. 
\end{itemize}

\noindent\textit{Center for Access to Engineering Excellence (CAEE)}                        	          
The Center for Access to Engineering Excellence (227 Bechtel Engineering Center;
\url{https://engineering.berkeley.edu/student-services/academic-support}) is an inclusive center that offers study spaces, nutritious snacks, and tutoring in >50 courses for Berkeley engineers and other majors across campus.  The Center also offers a wide range of professional development, leadership, and wellness programs, and loans clickers, laptops, and professional attire for interviews. \\
 
\noindent\textit{Disabled Students' Program (DSP)}
The Disabled Student's Program (260 C\'{e}sar Ch\'{a}vez Student Center \#4250; 510-642-0518;  \url{http://dsp.berkeley.edu}) serves students with disabilities of all kinds. Services are individually designed and based on the specific needs of each student as identified by DSP's Specialists.\\
 
\noindent\textit{Counseling and Psychological Services}    	 
The main University Health Services Counseling and Psychological Services staff is located at the Tang Center (\url{http://uhs.berkeley.edu}; 2222 Bancroft Way; 642-9494) and provides confidential assistance to students managing problems that can emerge from illness such as financial, academic, legal, family concerns, and more.\\
 
\noindent To improve access for engineering students, a licensed psychologist from the Tang Center also holds walk-in appointments for confidential counseling in 241 Bechtel Engineering Center (check here for schedule: \url{https://engineering.berkeley.edu/student-services/advising-counseling}). \\
 
\noindent\textit{The Care Line (PATH to Care Center)}
The Care Line (510-643-2005; \url{https://care.berkeley.edu/care-line/}) is a 24/7, confidential, free, campus-based resource for urgent support around sexual assault, sexual harassment, interpersonal violence, stalking, and invasion of sexual privacy. The Care Line will connect you with a confidential advocate for trauma-informed crisis support including time-sensitive information, securing urgent safety resources, and accompaniment to medical care or reporting.\\
 
\noindent\textit{Ombudsperson for Students}                                   	             
The Ombudsperson for Students (102 Sproul Hall; 642-5754; \url{http://students.berkeley.edu/Ombuds})  provides a confidential service for students involved in a University-related problem (academic or administrative), acting as a neutral complaint resolver and not as an advocate for any of the parties involved in a dispute. The Ombudsman can provide information on policies and procedures affecting students, facilitate students' contact with services able to assist in resolving the problem, and assist students in complaints concerning improper application of University policies or procedures. All matters referred to this office are held in strict confidence. The only exceptions, at the sole discretion of the Ombudsman, are cases where there appears to be imminent threat of serious harm.\\
 
\noindent\textit{UC Berkeley Food Pantry}                   
The UC Berkeley Food Pantry (\#68 Martin Luther King Student Union; \url{https://pantry.berkeley.edu}) aims to reduce food insecurity among students and staff at UC Berkeley, especially the lack of nutritious food. Students and staff can visit the pantry as many times as they need and take as much as they need while being mindful that it is a shared resource. The pantry operates on a self-assessed need basis; there are no eligibility requirements.  The pantry is not for students and staff who need supplemental snacking food, but rather, core food support.\\


\noindent \textbf{HW 0}: by sept 2 submit on bcourses
1.	What is your favorite activity outside of class?
2.	What is one principle of engagement from the code of conduct?


\end{document}
