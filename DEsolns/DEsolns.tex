%--------------------------------------------------------------------
% NE 155 (intro to numerical simulation of radiation transport)
% Spring 2014

% formatting
\documentclass[12pt]{article}
\usepackage[top=1in, bottom=1in, left=1in, right=1in]{geometry}

\usepackage{setspace}
\onehalfspacing

\setlength{\parindent}{0mm} \setlength{\parskip}{1em}


% packages
\usepackage{amssymb}
%% The amsthm package provides extended theorem environments
\usepackage{amsthm}
\usepackage{epsfig}
\usepackage{times}
\renewcommand{\ttdefault}{cmtt}
\usepackage{amsmath}
\usepackage{graphicx} % for graphics files

% Draw figures yourself
\usepackage{tikz} 

% The float package HAS to load before hyperref
\usepackage{float} % for psuedocode formatting
\usepackage{xspace}

% from Denovo methods manual
\usepackage{mathrsfs}
\usepackage[mathcal]{euscript}
\usepackage{color}
\usepackage{array}

\usepackage[pdftex]{hyperref}

\newcommand{\nth}{n\ensuremath{^{\text{th}}} }
\newcommand{\ve}[1]{\ensuremath{\mathbf{#1}}}
\newcommand{\macro}{\ensuremath{\Sigma}}
\newcommand{\vOmega}{\ensuremath{\hat{\Omega}}}

\newcommand{\cc}[1]{\ensuremath{\overline{#1}}}
\newcommand{\ccm}[1]{\ensuremath{\overline{\mathbf{#1}}}}


%--------------------------------------------------------------------
%--------------------------------------------------------------------
\begin{document}
\begin{center}
{\bf NE 155, Classes 21-23, S14 \\
Finite Difference Methods for the DE \\ March 12-17, 2014}
\end{center}

\setlength{\unitlength}{1in}
\begin{picture}(6,.1) 
\put(0,0) {\line(1,0){6.25}}         
\end{picture}

%--------------------------------------------------------------------
\section{Finite-Difference Method}
\underline{Problem} Consider the following second order ODE:
\[f''(x) = p(x)f'(x) + q(x)f(x) + r(x)\]
%
defined on a segment $[a,b]$ with $f(a) = \alpha$ and $f(b) = \beta$ (a boundary value problem).  

Now, let's spatially discretize the equation:
%
\begin{center}
\begin{tikzpicture}
\draw (-.25,0)--(1.25,0);
\draw[dotted] (1.25,0)--(2.75,0);
\draw (2.75,0)--(5.25,0);
\draw[dotted] (5.25,0)--(6.75,0);
\draw (6.75,0)--(8.25,0);
%\draw (4,0)--(5.25,0);
\draw (0,-.25)--(0,.25);
\draw (1,-.25)--(1,.25);
%\draw (2,-.25)--(2,.25);
\draw (3,-.25)--(3,.25);
\draw (4,-.25)--(4,.25);
\draw (5,-.25)--(5,.25);
\draw (7,-.25)--(7,.25);
\draw (8,-.25)--(8,.25);
\node[below] at (0,-.25) {$x_0$};
\node[below] at (1,-.25) {$x_1$};
\node[below] at (3,-.25) {$x_{i-1}$};
\node[below] at (4,-.25) {$x_i$};
\node[below] at (5,-.25) {$x_{i+1}$};
\node[below] at (7,-.25) {$x_{n-1}$};
\node[below] at (8,-.25) {$x_n$};
\node[above] at (0.5, 0.5) {$h$};
\node[above] at (3.5, 0.5) {$h$};
\end{tikzpicture}
\end{center}
%
where $x_0 = a$, $x_n = b$, and $h$ is the mesh spacing. There are $n+1$ points and $n$ mesh cells.

We can use \textbf{central difference} to approximate the derivatives on this grid. Let's use the $O(h^2)$ versions:
%
\begin{align}
f'(x_i) &= \frac{f(x_i + h) - f(x_i - h)}{2h} - \frac{h^2}{6} f'''(\mu) \nonumber \\
%
f''(x_i) &= \frac{f(x_i + h) - 2f(x_i) + f(x_i - h)}{h^2} + \frac{h^2}{12}f^{(4)}(\mu) \nonumber
\end{align}
%
We will also define $p_i = p(x_i)$, $q_i = q(x_i)$, $r_i = r(x_i)$.

Substituting into the original equation, we get:
\begin{align}
\frac{f_{i+1} - 2f_i + f_{i-1}}{h^2} &= p_i \frac{f_{i+1} - f_{i-1}}{2h} + q_i f_i + r_i \qquad i = 1, 2, \dots, n-1 \nonumber \\
%
\bigl(\frac{-h}{2}p_i - 1\bigr) f_{i-1} &+ \bigl(2 + h^2q_i\bigr)f_i + \bigl(\frac{h}{2}p_i - 1\bigr) f_{i+1} = -h^2 r_i \qquad i = 1, 2, \dots, n-1 \nonumber
\end{align}

We only have $n-1$ equations, but because the boundaries are fixed that is all we need. This is clear when we look at the $i=1$ and $i=n-1$ cases:
\begin{align}
&\bigl(2 + h^2q_1\bigr)f_1 + \bigl(\frac{h}{2}p_1 - 1\bigr) f_{2} = -h^2 r_1 + \underbrace{\bigl(\frac{h}{2}p_1 + 1\bigr) \underbrace{\alpha}_{f_0}}_{bc_L} \nonumber \\
 %
&\bigl(\frac{-h}{2}p_{n-1} - 1\bigr) f_{n-2} + \bigl(2 + h^2q_{n-1}\bigr)f_{n-1} = -h^2 r_{n-1} + \underbrace{\bigl(\frac{-h}{2}p_{n-1} + 1\bigr) \underbrace{\beta}_{f_n}}_{bc_R} \nonumber
\end{align}

Thus, we can write this as a matrix equation:
\begin{equation}
\begin{pmatrix}
\bigl(2 + h^2q_1\bigr) & \bigl(\frac{h}{2}p_1 - 1\bigr) & & \hdots & 0 \\
%
\bigl(\frac{-h}{2}p_2 - 1\bigr) & \bigl(2 + h^2q_2\bigr) & \bigl(\frac{h}{2}p_2 - 1\bigr) & &  \\
\vdots & \ddots & \ddots & \ddots & \vdots \\
& & \bigl(\frac{-h}{2}p_{n-2} - 1\bigr) & \bigl(2 + h^2q_{n-2}\bigr)  & \bigl(\frac{h}{2}p_{n-2} - 1\bigr) \\
0 & \hdots & & \bigl(\frac{-h}{2}p_{n-1} - 1\bigr) & \bigl(2 + h^2q_{n-1}\bigr) \\
\end{pmatrix}
%
\begin{pmatrix}
f_1 \\ f_2 \\ \vdots \\ f_{n-2} \\ f_{n-1} \\
\end{pmatrix}
= 
\begin{pmatrix}
-h^2 r_1 + bc_L \\ -h^2 r_2 \\ \vdots \\ -h^2 r_{n-1} \\ -h^2 r_{n-1} + bc_R\\ 
\end{pmatrix}\nonumber
\end{equation}

%
%\subsubsection{Two Variables}
%\begin{center}
%\begin{tikzpicture}
%\draw (-.25,0)--(7.25,0);
%\draw (1,-.25)--(1,.25);
%\draw (3,-.25)--(3,5.25);
%\draw (4,-.25)--(4,5.25);
%\draw (5,-.25)--(5,5.25);
%\draw (7,-.25)--(7,.25);
%\node[below] at (0,-.25) {$x_0$};
%\node[below] at (1,-.25) {$x_1$};
%\node[below] at (3,-.25) {$x_{i-1}$};
%\node[below] at (4,-.25) {$x_i$};
%\node[below] at (5,-.25) {$x_{i+1}$};
%\node[below] at (7,-.25) {$x_n$};
%\node at (3.5, 2) {$\Delta x$};
%% begin y
%\draw (0,-.25)--(0,7.25);
%\draw (-.25,1)--(.25,1);
%\draw (-.25,3)--(5.25,3);
%\draw (-.25,4)--(5.25,4);
%\draw (-.25,5)--(5.25,5);
%\draw (-.25,7)--(.25,7);
%\node[left] at (-.25, 0) {$y_0$};
%\node[left] at (-.25,1) {$y_1$};
%\node[left] at (-.25,3) {$y_{i-1}$};
%\node[left] at (-.25,4) {$y_i$};
%\node[left] at (-.25,5) {$y_{i+1}$};
%\node[left] at (-.25,7) {$y_n$};
%\node at (2,3.5) {$\Delta y$};
%\end{tikzpicture}
%\end{center}
%
%\begin{align}
%\frac{\partial^2 f_{i,j}}{\partial x^2} = \frac{\partial}{\partial x}\bigl(\frac{\partial f_{i,j}}{\partial x}\bigr) =
%\frac{f_{i-1,j} - 2f_{i,j} + f_{i+1,j}}{\Delta x^2} \\
%%
%\frac{\partial^2 f_{i,j}}{\partial y^2} = \frac{\partial}{\partial y}\bigl(\frac{\partial f_{i,j}}{\partial y}\bigr) =
%\frac{f_{i,j-1} - 2f_{i,j} + f_{i,j+1}}{\Delta y^2}
%\end{align}


%--------------------------------------------------------------------
%--------------------------------------------------------------------
%\bibliographystyle{plain}
%\bibliography{LinearSolns} 

\end{document}