\NeedsTeXFormat{LaTeX2e}
\documentclass[a4paper, 12 pt]{curve}
%\documentclass[12pt]{article}
%\textwidth=7in
%\textheight=9.5in
%\topmargin=-1in
%\headheight=0in
%\headsep=.5in
%\hoffset=-.85in

%\usepackage[cm]{fullpage}
\usepackage[top=0.75in, bottom=0.75in, left=1in, right=1in]{geometry}
\pagestyle{empty}
\usepackage{tabu}
\parindent 0ex
\usepackage{hyperref}

\renewcommand{\thefootnote}{\fnsymbol{footnote}}
\begin{document}

\begin{center}
{\bf NE 155\\ Final Project Rubric
}
\end{center}

\setlength{\unitlength}{1in}
\begin{picture}(6,.1) 
\put(0,0) {\line(1,0){6.25}}         
\end{picture}

\renewcommand{\arraystretch}{2}

\underline{\textbf{Final report}}: should be $\sim$5 pages with \textbf{1.5 spacing} per team member. This may vary based on the specific project; please use your best judgment. 
\vspace*{1em}\\
Please include these items as clearly labeled sections
%
\begin{enumerate}
\item \textbf{Introduction:} What does the code you wrote do? Also, preview what you are going to talk about.

\item \textbf{Mathematics:} Write the continuous and discretized equations that you are solving, defining all terms. Include any derivations needed to reach discretized equations as applicable (can be included an the appendix if they are lengthy).

\item \textbf{Algorithms:} Include the algorithms that you implemented in your code.

\item \textbf{Code Use:} Describe how to use your code, including inputs needed and output expected. 

\item \textbf{Test Problems and Results:} Describe any testing you did to demonstrate your code is correct and present any results from test problems.

\item \textbf{References:} You must have references that you cite in your paper (including the interim report).
\end{enumerate}
\underline{\textbf{Products}}: You must submit your code and at least one example input and corresponding output (I strongly encourage you to version control your code and submit access to the repository). Part of the project grade will be based on whether the code executes properly.

\vspace*{2em}
\underline{\textbf{Interim report}}: please replace the ``Test Problems and Results" section (and possibly also ``Code Use" depending on how far you've gotten) with \textbf{Plans for Completion} and keep in mind that these plans should include plans for what tests/inputs you will give to your code. The first four sections don't have to be completely polished, but they should at least be very solid drafts. I will provide feedback, so the better they are when I read them the more useful the feedback will be. 
\vspace*{1em}\\
This should be a maximum of 4, 6, or 8 pages with 1.5 spacing for 1, 2, or 3 people, respectively (varying depending on compactness of mathematics, algorithms, etc.).

\vspace*{2em}
\underline{\textbf{Final Presentation}}: please include these same items as the final paper in the final presentation. If you are using equations and algorithms we learned in class, just mention you used what we did in class and spend more time on how you implemented the code and how you tested it. The presentations should be approximately 6, 9, or 12 minutes for 1, 2, or 3 people, respectively.  This may vary based on the specific project; please use your best judgment. Part of the final presentation should be a code execution demonstration.

\vspace*{2em}
\input{notes}

\vspace*{2em}
I will use the following rubric for evaluating the paper:
\begin{center}
\begin{tabu}{| X | c | c |}\hline
\textbf{Category} & \textbf{Possible Points} & \textbf{Earned Points} \\ \hline \hline
Math, discretizations, and algorithms are correct & 10 & \\ \hline
Code input, output, and tests are well-designed & 10 & \\ \hline
Work implemented correctly & 10 & \\ \hline
Appropriate logical flow of paper & 3 & \\ \hline
Complete sentences; correct grammar and spelling & 7 & \\ \hline
Sources properly documented & 5 & \\ \hline
Total & 45 & \\\hline
\end{tabu} 
\end{center}

\vspace*{1 em}
and for the presentation:
\begin{center}
\begin{tabu}{| X | c | c |}\hline
\textbf{Category} & \textbf{Possible Points} & \textbf{Earned Points} \\ \hline \hline
Explanation of implemented code (math, discretizations, algorithms) is clear & 6 & \\ \hline
Code demonstration works and is understandable & 6 & \\ \hline
Good presentation skills: eye contact, volume, clarity of slides, etc. & 6 & \\ \hline
Appropriate presentation length & 2 & \\ \hline
Total & 20 & \\\hline
\end{tabu} 
\end{center}


\end{document}