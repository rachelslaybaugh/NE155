\documentclass[12pt]{article}
%\textwidth=7in
%\textheight=9.5in
%\topmargin=-1in
%\headheight=0in
%\headsep=.5in
%\hoffset=-.85in

%\usepackage[cm]{fullpage}
\usepackage[top=0.75in, bottom=0.75in, left=1in, right=1in]{geometry}
\pagestyle{empty}
\usepackage{tabu}
\usepackage{hyperref}
\usepackage{tabu}

\renewcommand{\thefootnote}{\fnsymbol{footnote}}
\begin{document}

\begin{center}
{\bf NE 155 - Introduction to Numerical Simulations in Radiation Transport \\ Final Project \\ Due May 13, 2021
}
\end{center}

\setlength{\unitlength}{1in}
\begin{picture}(6,.1) 
\put(0,0) {\line(1,0){6.25}}         
\end{picture}

\renewcommand{\arraystretch}{2}

Below is a list of possible projects (there are Monte Carlo and deterministic options). If you intend to share a project among a team of students (maximum three students per team), check to ensure that the project has sufficient scope. The project is 40\% of your grade and is due on May 13 at the time of the final presentation. The following schedule will be imposed:

\vspace*{2 em}
\textbf{March 30:} Decide which project to work on; turn in list of team members (if applicable) and a one- or two-page \textit{abstract}, including:
\begin{enumerate}
\item What you plan to do
\item Major steps to execute the project
\item Deadlines associated with each step
\item What you need to do to accomplish each step (laying out a path to success)
\item If in a team, the division of work
\end{enumerate}

\vspace*{2 em}
\textbf{April 15:} Submit a written \textit{interim} report (4, 6, or 8 pages maximum for 1, 2, or 3 people, respectively) explaining your project. See the rubric for details of what to include.

\vspace*{2 em}
\textbf{May 13:} Presentations (between 6 and 12 minutes, depending on project and team size). See the rubric for details of what to include.

\vspace*{2 em}
\textbf{May 13:} Final written reports (about 6-7 pages/team member as a rule of thumb) are due. See the rubric for details of what to include.

\begin{center}
\begin{tabu}{| l | c |}\hline
\textbf{Item} & \textbf{Relative Points} \\ \hline \hline
abstract & 10 \\
interim report & 25 \\
final report & 45 \\
final presentation & 20 \\ \hline
\end{tabu} 
\end{center}

\clearpage 
\begin{center}
\textbf{Potential project topics:}
\end{center}

\begin{enumerate}
\item Write a 2D diffusion solver that has vacuum boundaries on the bottom and left faces and reflecting boundaries on the top and right boundaries. I have more detailed specifications and some helpful tasks to facilitate completion if you choose this project. 

\item Write a 2D transport solver that has vacuum boundaries on the bottom and left faces and reflecting boundaries on the top and right boundaries. I have more detailed specifications and some helpful tasks to facilitate completion if you choose this project.

\item Write a simple Monte Carlo solver that will include tallies, a subset of sampling routines, and implicit capture. I have more detailed specifications and some helpful tasks to facilitate completion if you choose this project. (It is unlikely this will be an easy choice if you do not already have familiarity with MC since we're covering that last).

\item Propose your own project; this can include writing a code or studying and characterizing the way a code performs. Please use this as an opportunity to expand on or compliment work you are already doing.

\item If you really really want to do an analysis project we can talk about that as an option.
\end{enumerate}

\vspace*{1 em}
PLEASE use version control for your project (git tutorial here: \url{http://bids.github.io/2016-01-14-berkeley/git/}). 

\vspace*{1 em}
If you are comfortable writing your project in Python, consider using PyNE (\url{http://pyne.io}) to facilitate your project. Depending on what you do, we might be able to contribute your project back to the PyNE code base over the summer.

\end{document}