\documentclass[12pt]{article}
%\textwidth=7in
%\textheight=9.5in
%\topmargin=-1in
%\headheight=0in
%\headsep=.5in
%\hoffset=-.85in

%\usepackage[cm]{fullpage}
\usepackage[top=0.75in, bottom=0.75in, left=1in, right=1in]{geometry}
\pagestyle{empty}
\usepackage{tabu}
\usepackage{url}

\setlength{\parindent}{0mm} \setlength{\parskip}{1em}

\renewcommand{\thefootnote}{\fnsymbol{footnote}}
\begin{document}

\begin{center}
{\bf NE 155 \\ 2D Diffusion Final Project Suggestions}
\end{center}

\setlength{\unitlength}{1in}
\begin{picture}(6,.1) 
\put(0,0) {\line(1,0){6.25}}         
\end{picture}

\renewcommand{\arraystretch}{2}

Write a 2D Diffusion solver that has vacuum boundaries on the bottom and left faces and reflecting boundaries on the top and right boundaries. Adapt the advice below as applicable based on whether you use a compiled language like C++ or an interpreted language like Python. 

Good programming practices (there are some tutorials for all of these things available through Berkeley's The Hacker Within as well as Software Carpentry.): 
\begin{itemize}
\item Please version control your code on something like GitHub. 
\item Consider writing code comments with something like Doxygen (\url{http://www.doxygen.org}) to document the API (application programming interface). 
\item Consider writing tests for your code using something like gtest (\url{https://code.google.com/\\p/googletest/}) - this might be a good extension for a two person project in particular.
\end{itemize}

Components that you'll need to include with the code:
\begin{itemize}
\item A README that states
  \begin{itemize}
  \item How to compile the code (if applicable)
  \item How to execute the code
  \item Status of the code (as applicable): operational/compiles but doesn't run/doesn't compile/known bugs
  \item Describe the problem solved by your code, expected input, resulting output, and any limitations or restrictions
  \end{itemize}
  
\item A sample input file and corresponding output file produced by your code.

\item Things you'll want to include in the input file:
  \begin{itemize}
  \item Number of x and y cells
  \item Cells size in each dimension (you can choose if you want it to be uniform or non-uniform)
  \item Number of materials
  \item A way to assign materials
  \item Physical constants for each material: $D$, $\Sigma_a$
  \item Fixed source term. You need to decide things like whether you want a source everywhere or only in some cells, whether it must be constant or can be a function of space, etc. and how to express those choices in the input file.
  \item If you want to allow for different boundary condition choices, how to specify those.
  \end{itemize}
  
\end{itemize}

The general framework for your code should be:
\begin{itemize}
\item \textit{Main Module:} calls subroutines listed in specific sequence, prints execution time, and terminates execution

\item \textit{Version data subroutine:} write code name, version number, author name(s), date and time of execution to an output file (perhaps also print to screen).

\item \textit{Input data subroutine:} read and/or process the input data. Check all input values for correctness/sensibility, e.g., all values are positive, array dimensions are correct, etc. Print an error message and terminate if one or more errors occur, otherwise print notification of successful input checking. 

\item \textit{Input echo subroutine:} print the input data for each cell to the output file (this is good for reproducibility). Please format this in some useful way.

\item \textit{Diffusion solver subroutine:} implement the discretized diffusion solver here. While building the surrounding structure you can just have this print ``will solve DE here". 

\end{itemize}

\textbf{Modifications to consider if you are working in a team.} You can choose some of these to add complexity, which would be appropriate for a team. 
\begin{itemize}
\item allow the option to choose which boundaries have which boundary conditions
\item allow  variable mesh size
\item allow more flexibility in how to assign the source
\item implement comments that can be easily made into an API
\item implement testing
\end{itemize}



\end{document}