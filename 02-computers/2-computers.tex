%%%%%%%%%%%%%%%%%%%%%%%%%%%%%%%%%%%%%%%%%%%%%%%%%%%%%%%%%%%%
%%  Class 1, NE 155
%%

\documentclass[xcolor=x11names,compress]{beamer}

\definecolor{CoolBlack}{rgb}{0.0, 0.18, 0.39}
%% General document %%%%%%%%%%%%%%%%%%%%%%%%%%%%%%%%%%
\usepackage{graphicx}
\usepackage{tikz}
\usetikzlibrary{decorations.fractals}
\usepackage{hyperref}
%%%%%%%%%%%%%%%%%%%%%%%%%%%%%%%%%%%%%%%%%%%%%%%%%%%%%%

%% Beamer Layout %%%%%%%%%%%%%%%%%%%%%%%%%%%%%%%%%%
\useoutertheme[subsection=false,shadow]{miniframes}
\useinnertheme{default}
\usefonttheme{serif}
\usepackage{palatino}
\usepackage{tabu}
% Links
\usepackage{hyperref}
\definecolor{links}{HTML}{003262}
\hypersetup{colorlinks,linkcolor=,urlcolor=links}

% addition of color
\usepackage{xcolor}
\definecolor{CoolBlack}{rgb}{0.0, 0.18, 0.39}
\definecolor{byellow}{rgb}{0.55037, 0.38821, 0.06142}
\definecolor{dgreen}{rgb}{0.,0.6,0.}
\definecolor{RawSienna}{cmyk}{0,0.72,1,0.45}
\definecolor{forestgreen(web)}{rgb}{0.13, 0.55, 0.13}
\definecolor{cardinal}{rgb}{0.77, 0.12, 0.23}

\setbeamerfont{title like}{shape=\scshape}
\setbeamerfont{frametitle}{shape=\scshape}

\setbeamercolor*{lower separation line head}{bg=CoolBlack}
\setbeamercolor*{normal text}{fg=black,bg=white}
\setbeamercolor*{alerted text}{fg=dgreen} % just testing; I think this looks better
\setbeamercolor*{example text}{fg=black}
\setbeamercolor*{structure}{fg=black}

\setbeamercolor*{palette tertiary}{fg=black,bg=black!10}
\setbeamercolor*{palette quaternary}{fg=black,bg=black!10}

% Margins
\usepackage{changepage}

\mode<presentation>
{
  \definecolor{berkeleyblue}{HTML}{003262}
  \definecolor{berkeleygold}{HTML}{FDB515}
  \usetheme{Boadilla}      % or try Darmstadt, Madrid, Warsaw, Boadilla...
  %\usecolortheme{dove} % or try albatross, beaver, crane, ...
  \setbeamercolor{structure}{fg=berkeleyblue,bg=berkeleygold}
  \setbeamercolor{palette primary}{bg=berkeleyblue,fg=white} % changed this
  \setbeamercolor{palette secondary}{fg=berkeleyblue,bg=berkeleygold} % changed this
  \setbeamercolor{palette tertiary}{bg=berkeleyblue,fg=white} % changed this
  \usefonttheme{structurebold}  % or try serif, structurebold, ...
  \useinnertheme{circles}
  \setbeamertemplate{navigation symbols}{}
  \setbeamertemplate{caption}[numbered]
  \usebackgroundtemplate{}
}
%---

\renewcommand{\(}{\begin{columns}}
\renewcommand{\)}{\end{columns}}
\newcommand{\<}[1]{\begin{column}{#1}}
\renewcommand{\>}{\end{column}}

% adding slide numbers
\addtobeamertemplate{navigation symbols}{}{%
    \usebeamerfont{footline}%
    \usebeamercolor[fg]{footline}%
    \hspace{1em}%
    \insertframenumber/\inserttotalframenumber
}

% equation stuff
\newcommand{\Macro}{\ensuremath{\Sigma}}
\newcommand{\Sn}{\ensuremath{S_N} }
\newcommand{\vOmega}{\ensuremath{\hat{\Omega}}}
\usepackage{mathrsfs}
\usepackage[mathcal]{euscript}
\usepackage{amssymb}
\usepackage{amsthm}
\usepackage{epsfig}
\usepackage{amsmath}

%%%%%%%%%%%%%%%%%%%%%%%%%%%%%%%%%%%%%%%%%%%%%%%%%%
% title stuff for footer
\title{NE 155}
\author{R.\ N.\ Slaybaugh}
\date{January 22, 2016}

\begin{document}

%%%%%%%%%%%%%%%%%%%%%%%%%%%%%%%%%%%%%%%%%%%%%%%%%%%%%%
%%%%%%%%%%%%%%%%%%%%%%%%%%%%%%%%%%%%%%%%%%%%%%%%%%%%%%
\begin{frame}
\title{NE 155\\Introduction to Numerical Simulations in Radiation Transport}
\subtitle{Lecture 2: Computing}
\titlepage
\end{frame}

%%%%%%%%%%%%%%%%%%%%%%%%%%%%%%%%%%%%%%%%%%%%%%%%%%%%%%
%%%%%%%%%%%%%%%%%%%%%%%%%%%%%%%%%%%%%%%%%%%%%%%%%%%%%%
\begin{frame}{Outline}
%\tableofcontents
\begin{enumerate}
\item History and terminology
%  - define flops as measure of what we care about
%  - chronological highlights of a few machines; show pictures, state details
\item Basic computer architecture
%  - hardware v software; we're talking about hardware now but the rest of the class will focus on software. We will care a little bit about the interface
%  - definitions
%  - Moore's law
\item Introduction to Parallelism
%  - basic considerations
%  - types of parallelism
%  - speedup
%  - increase in use and potential
%  - current status
\item Current supercomputers
%  - Moore's law continuation?
%  - GPUs
%  - mic chips
%  - What happens may have a large impact on algorithms
\end{enumerate}
\end{frame}

%%%%%%%%%%%%%%%%%%%%%%%%%%%%%%%%%%%%%%%%%%%%%%%%%%%%%%
%%%%%%%%%%%%%%%%%%%%%%%%%%%%%%%%%%%%%%%%%%%%%%%%%%%%%%
\begin{frame}{How do we measure utility?\footnote{\href{en.wikipedia.org}{en.wikipedia.org}}}

\textbf{IPS} (Instructions Per Second) is a measure of a computer's processor speed. IPS can be useful when comparing performance between processors made from a similar architecture, but are difficult to compare between CPU architectures

\vspace*{1 em}
\textbf{Clock rate} typically refers to the frequency at which a CPU is running. It is measured in the SI unit Hertz.

\vspace*{1 em}
\textbf{FLOPS} (FLoating-point Operations Per Second) is a measure of computer performance, useful in fields of scientific calculations that make heavy use of floating-point calculations. 
\end{frame}

%%%%%%%%%%%%%%%%%%%%%%%%%%%%%%%%%%%%%%%%%%%%%%%%%%%%%%
%%%%%%%%%%%%%%%%%%%%%%%%%%%%%%%%%%%%%%%%%%%%%%%%%%%%%%
\section{\scshape History}
\subsection{History}
\begin{frame}{Computing Machines, Origins}
\begin{figure}
\includegraphics[height=2.25in,clip]{RomanAbacus}
\caption{Roman Abacus, \href{http://history-computer.com/CalculatingTools/abacus.html}{http://history-computer.com/CalculatingTools/abacus.html}}
\end{figure}

\end{frame}

%%%%%%%%%%%%%%%%%%%%%%%%%%%%%%%%%%%%%%%%%%%%%%%%%%%%%%
%%%%%%%%%%%%%%%%%%%%%%%%%%%%%%%%%%%%%%%%%%%%%%%%%%%%%%
\begin{frame}{Early Development of Computing}

\begin{figure}
\includegraphics[height=2in,clip]{BabbageDiffMachine}
\caption{Reconstruction of Babbage's Analytical Engine, the first general-purpose programmable computer, \href{http://en.wikipedia.org/wiki/History_of_computing_hardware
\#Punched_card_data_processing}{http://en.wikipedia.org/wiki/History\_of\_computing\_hardware
\#Punched\_card\_data\_processing}}
\end{figure}

\end{frame}

%%%%%%%%%%%%%%%%%%%%%%%%%%%%%%%%%%%%%%%%%%%%%%%%%%%%%%
%%%%%%%%%%%%%%%%%%%%%%%%%%%%%%%%%%%%%%%%%%%%%%%%%%%%%%
\begin{frame}{First Electromechanical Computers}

\begin{figure}
\includegraphics[height=2in,clip]{Z3DeutschesMuseum}
\caption{Zuse Z3 replica on display at Deutsches Museum in Munich, \href{en.wikipedia.org/wiki/Z3_(computer)}{en.wikipedia.org/wiki/Z3\_(computer)}}
\end{figure}

\end{frame}

%%%%%%%%%%%%%%%%%%%%%%%%%%%%%%%%%%%%%%%%%%%%%%%%%%%%%%
%%%%%%%%%%%%%%%%%%%%%%%%%%%%%%%%%%%%%%%%%%%%%%%%%%%%%%
\begin{frame}{Stored Programs}

\begin{figure}
\includegraphics[height=2in,clip]{ManchesterMark1}
\caption{The Manchester Mark 1 was one of the world's first stored-program computers, \href{http://www.computer50.org/mark1/ip-mm1.mark1.html}{http://www.computer50.org/mark1/ip-mm1.mark1.html}}
\end{figure}

\end{frame}

%%%%%%%%%%%%%%%%%%%%%%%%%%%%%%%%%%%%%%%%%%%%%%%%%%%%%%
%%%%%%%%%%%%%%%%%%%%%%%%%%%%%%%%%%%%%%%%%%%%%%%%%%%%%%
\begin{frame}{Microprogramming, Magnetic Storage, Transistors}
\begin{itemize}
\item 1951: realization that CPUs can be controlled by a miniature, highly specialized computer program in high-speed ROM \vspace*{0.5 em}
\item 1954: magnetic core memory was rapidly displacing most other forms of temporary storage \vspace*{0.5 em}
\item 1956: IBM introduced the first disk storage unit: using 50 24-inch metal disks, it stored 5 MB of data for \$10,000 per MB (\$90,000 in 2014 \$s)\vspace*{0.5 em}
\item 1947: invention of the bipolar transistor; this replaced vacuum tubes by 1955 $\rightarrow$ ``Second Generation" of computer designs
\end{itemize}
\end{frame}

%%%%%%%%%%%%%%%%%%%%%%%%%%%%%%%%%%%%%%%%%%%%%%%%%%%%%%
%%%%%%%%%%%%%%%%%%%%%%%%%%%%%%%%%%%%%%%%%%%%%%%%%%%%%%
\begin{frame}{Supercomputers}

\begin{figure}
\includegraphics[height=2in,clip]{Atlas1963}
\caption{The University of Manchester Atlas 1963, \href{http://en.wikipedia.org/wiki/History_of_computing_hardware
\#Punched_card_data_processing}{http://en.wikipedia.org/wiki/History\_of\_computing\_hardware
\#Punched\_card\_data\_processing}}
\end{figure}

\end{frame}

%%%%%%%%%%%%%%%%%%%%%%%%%%%%%%%%%%%%%%%%%%%%%%%%%%%%%%
%%%%%%%%%%%%%%%%%%%%%%%%%%%%%%%%%%%%%%%%%%%%%%%%%%%%%%
\begin{frame}{Integrated Circuit}
With the advent of the \alert{transistor} and the work on semi-conductors generally, it now seems possible to envisage electronic equipment in a solid block with no connecting wires. The block may consist of layers of insulating, conducting, rectifying and amplifying materials, the \alert{electronic functions being connected directly} by cutting out areas of the various layers.\\
\vspace*{1.5 em}
\hspace*{0.5 in}Geoffrey W.A. Dummer, Royal Radar Establishment of the Ministry of Defence, 1952

\end{frame}

%%%%%%%%%%%%%%%%%%%%%%%%%%%%%%%%%%%%%%%%%%%%%%%%%%%%%%
%%%%%%%%%%%%%%%%%%%%%%%%%%%%%%%%%%%%%%%%%%%%%%%%%%%%%%
\begin{frame}{Generations 4-6}
\begin{itemize}
\item Very large scale integration of devices on chip \vspace*{0.5 em}
\item C and FORTRAN programming languages\vspace*{0.5 em}
\item \alert{UNIX} operating system (Bell labs, Berkeley)\vspace*{0.5 em}
\item Large scale parallel processing; supercomputing centers\vspace*{0.5 em}
\item Shared and distributed memory\vspace*{0.5 em}
\item Parallel/vector shared/distributed memory combinations\vspace*{0.5 em}
\item High speed networking
\end{itemize}
\end{frame}

%%%%%%%%%%%%%%%%%%%%%%%%%%%%%%%%%%%%%%%%%%%%%%%%%%%%%%
%%%%%%%%%%%%%%%%%%%%%%%%%%%%%%%%%%%%%%%%%%%%%%%%%%%%%%
\section{\scshape Computer Architecture}
\subsection{Computer Architecture}
\begin{frame}{Computer Architecture}
 \begin{figure}
   \begin{center}
     \includegraphics[height=2.25in,clip]{ComputerArchitecture}
   \end{center}
 \end{figure}
\end{frame}

%%%%%%%%%%%%%%%%%%%%%%%%%%%%%%%%%%%%%%%%%%%%%%%%%%%%%%
%%%%%%%%%%%%%%%%%%%%%%%%%%%%%%%%%%%%%%%%%%%%%%%%%%%%%%
\begin{frame}{Components}
 \begin{figure}
   \begin{center}
     \includegraphics[height=2.25in,clip]{Components}
   \end{center}
 \end{figure}
\end{frame}

%%%%%%%%%%%%%%%%%%%%%%%%%%%%%%%%%%%%%%%%%%%%%%%%%%%%%%
%%%%%%%%%%%%%%%%%%%%%%%%%%%%%%%%%%%%%%%%%%%%%%%%%%%%%%
%\begin{frame}{Types of Memory}
%\begin{itemize}
%\item \textbf{Read-only memory} (ROM): storage medium in which data cannot be modified, so it is mainly used to distribute firmware 
%\item \textbf{Random Access Memory} (RAM): allows stored data to be accessed directly in any random order\footnote{Many computer systems have a memory hierarchy consisting different systems referred to collectively as ``RAM". The various subsystems can have very different access times.}
%\item Other data storage media only read and write data consecutively (time impact)
%\end{itemize}
%\end{frame}

%%%%%%%%%%%%%%%%%%%%%%%%%%%%%%%%%%%%%%%%%%%%%%%%%%%%%%
%%%%%%%%%%%%%%%%%%%%%%%%%%%%%%%%%%%%%%%%%%%%%%%%%%%%%%
\begin{frame}{Moore's ``Law"}
The number of transistors on integrated circuits doubles approximately every 18 months. 

\begin{columns}
  \begin{column}{0.7\textwidth}
\begin{itemize}
\item In 1965 Gordon E. Moore described this trend and predicted it to hold for at least 10 years\vspace*{0.5 em}
\item He worked at Intel\vspace*{0.5 em}
\item The trend has held, but is expected to change around now...
\end{itemize}
  \end{column}
  \begin{column}{0.3\textwidth}
    \includegraphics[height=1.5in,clip]{GordonMoore1975}
  \end{column}
\end{columns}
\end{frame}

%%%%%%%%%%%%%%%%%%%%%%%%%%%%%%%%%%%%%%%%%%%%%%%%%%%%%%
%%%%%%%%%%%%%%%%%%%%%%%%%%%%%%%%%%%%%%%%%%%%%%%%%%%%%%
\begin{frame}{Moore's ``Law"}

\begin{figure}
\includegraphics[height=2.75in,clip]{CPU-Scaling}
%\caption{CPU scaling showing transistor density, power consumption, and efficiency, http://www.extremetech.com/computing/116561-the-death-of-cpu-scaling-from-one-core-to-many-and-why-were-still-stuck}
\end{figure}

\end{frame}

%%%%%%%%%%%%%%%%%%%%%%%%%%%%%%%%%%%%%%%%%%%%%%%%%%%%%%
%%%%%%%%%%%%%%%%%%%%%%%%%%%%%%%%%%%%%%%%%%%%%%%%%%%%%%
\section{\scshape Parallelism}
\subsection{Parallelism}
\begin{frame}{What is Parallel Architecture?}
A \textcolor{dgreen}{parallel computer} is a collection of processing elements that cooperate to solve large problems
quickly

\begin{itemize}
\item Resource allocation
  \begin{itemize}
  \item How much \textcolor{dgreen}{memory}?
  \item How \textcolor{dgreen}{many} elements?
  \item How \textcolor{dgreen}{powerful} are the elements?
  \end{itemize}
\item Data access, Communication, and Synchronization
  \begin{itemize}
  \item How do the elements cooperate and \textcolor{dgreen}{communicate}?
  \item How are \textcolor{dgreen}{data transmitted} between processors?
  \item What are the \textcolor{dgreen}{abstractions} and primitives for cooperation?
  \end{itemize}
\item Performance and Scalability
  \begin{itemize}
  \item How does it all translate into \textcolor{dgreen}{performance}?
  \item How does it \textcolor{dgreen}{scale}?
  \end{itemize}
\end{itemize}

\end{frame}

%%%%%%%%%%%%%%%%%%%%%%%%%%%%%%%%%%%%%%%%%%%%%%%%%%%%%%
%%%%%%%%%%%%%%%%%%%%%%%%%%%%%%%%%%%%%%%%%%%%%%%%%%%%%%
\begin{frame}{Forms of Parallelism}
\begin{itemize}
\item \textbf{Bit level}: increases in word size reduced the \# of instructions the processor needs
\item \textbf{Instruction level}: hardware and/or software perform operations simultaneously when possible
\item \textbf{Memory system}: overlap of memory operations with computation
\item \textbf{Operating system}: multiple jobs run in parallel on commodity symmetric multiprocessors (SMPs) 
\end{itemize}
There are limitations to all of these

\vspace*{1 em}
To achieve high performance, the programmer needs to identify, schedule, and coordinate parallel tasks and data

\end{frame}

%%%%%%%%%%%%%%%%%%%%%%%%%%%%%%%%%%%%%%%%%%%%%%%%%%%%%%
%%%%%%%%%%%%%%%%%%%%%%%%%%%%%%%%%%%%%%%%%%%%%%%%%%%%%%
\begin{frame}{Performance}

\begin{columns}
  \begin{column}{0.65\textwidth}
\begin{itemize}
\item \textbf{Strong Scaling}: increase element count with problem size fixed (solve a problem faster)\\
\vspace*{0.5 em}
$Speedup = \frac{\text{Time with P cores}}{\text{Time with 1 core}}$
\vspace*{1 em}
\item \textbf{Weak Scaling}: increase element count and problem size to keep problem size per element fixed (solve bigger problems)\\
\vspace*{0.5 em}
$Speedup = \frac{\text{Time with 1 core}}{\text{Time with P cores}}$
\end{itemize}
  \end{column}
  \begin{column}{0.35\textwidth}
    \includegraphics[height=2in,clip]{PerformanceGap}
  \end{column}
\end{columns}

\end{frame}

%%%%%%%%%%%%%%%%%%%%%%%%%%%%%%%%%%%%%%%%%%%%%%%%%%%%%%
%%%%%%%%%%%%%%%%%%%%%%%%%%%%%%%%%%%%%%%%%%%%%%%%%%%%%%
\begin{frame}{Types of Memory}

%\begin{itemize}
%\item \textbf{Shared Memory}: single memory space used by all processors; processors do not have to know where data resides
%\item \textbf{Distributed Memory}: each processor has its own private memory; tasks can only operate on local data; communication with a remote processor to get remote data
%\item \textbf{Distributed Shared Memory}: each node of a cluster has access to a large shared memory in addition to each node's limited non-shared private memory.
%\end{itemize}
\begin{center}
Shared \hspace*{2 em} Distributed \hspace*{2 em} Distributed Shared
\includegraphics[height=2.5in,clip]{MemorySystems}
\end{center}

\end{frame}

%%%%%%%%%%%%%%%%%%%%%%%%%%%%%%%%%%%%%%%%%%%%%%%%%%%%%%
%%%%%%%%%%%%%%%%%%%%%%%%%%%%%%%%%%%%%%%%%%%%%%%%%%%%%%
\begin{frame}{GPUs}
\begin{itemize}
\item Graphics Processing Unit (\textbf{GPU}): highly parallel, good for processing large blocks of data
\item General Purpose GPU (\textbf{GPGPU}): Using a GPU to do CPU work - computational science instead of graphics
\end{itemize}

\begin{center}
NVIDIA GPU \\ \vspace*{1 em}
\includegraphics[height=1.5in,clip]{GPU}
\end{center}
\end{frame}

%%%%%%%%%%%%%%%%%%%%%%%%%%%%%%%%%%%%%%%%%%%%%%%%%%%%%%
%%%%%%%%%%%%%%%%%%%%%%%%%%%%%%%%%%%%%%%%%%%%%%%%%%%%%%
\begin{frame}{MICs}
\begin{itemize}
\item  Many Integrated Core (\textbf{MIC}): combines many CPU cores onto a single chip
\item \textbf{Heterogeneous} architecture: GPUs + CPUs or MICs + CPUs, etc.
\end{itemize}

\begin{center}
Intel Xeon Phi Board\\ \vspace*{1 em}
\includegraphics[height=1.75in,clip]{Intel-Xeon-Phi-Board}
\end{center}
\end{frame}

%%%%%%%%%%%%%%%%%%%%%%%%%%%%%%%%%%%%%%%%%%%%%%%%%%%%%%
%%%%%%%%%%%%%%%%%%%%%%%%%%%%%%%%%%%%%%%%%%%%%%%%%%%%%%
\section{\scshape Supercomputing}
\subsection{Supercomputing}
\begin{frame}{Scientific Computing Demand}

\begin{center}
\includegraphics[height=2.75in,clip]{MemoryNeeds}
\end{center}

\end{frame}

%%%%%%%%%%%%%%%%%%%%%%%%%%%%%%%%%%%%%%%%%%%%%%%%%%%%%%
%%%%%%%%%%%%%%%%%%%%%%%%%%%%%%%%%%%%%%%%%%%%%%%%%%%%%%
\begin{frame}{\href{http://www.top500.org/}{Top 500 Computers, Nov 2015}}

\begin{center}
\includegraphics[height=2.5in]{Supercomputers-history-2015}\\
x axis in years
\end{center}

\end{frame}

%%%%%%%%%%%%%%%%%%%%%%%%%%%%%%%%%%%%%%%%%%%%%%%%%%%%%%
%%%%%%%%%%%%%%%%%%%%%%%%%%%%%%%%%%%%%%%%%%%%%%%%%%%%%%
\begin{frame}{\href{http://en.wikipedia.org/wiki/TOP500}{Top 10 Computers, Nov 2015}}

\begin{center}
\includegraphics[height=3in]{2015-top-10}
\end{center}

\end{frame}

%%%%%%%%%%%%%%%%%%%%%%%%%%%%%%%%%%%%%%%%%%%%%%%%%%%%%%
%%%%%%%%%%%%%%%%%%%%%%%%%%%%%%%%%%%%%%%%%%%%%%%%%%%%%%
\begin{frame}{Power Consumption, Nov 2013}

\begin{center}
\includegraphics[height=2.25in]{Top500-power}
\end{center}

\end{frame}

%%%%%%%%%%%%%%%%%%%%%%%%%%%%%%%%%%%%%%%%%%%%%%%%%%%%%%
%%%%%%%%%%%%%%%%%%%%%%%%%%%%%%%%%%%%%%%%%%%%%%%%%%%%%%
\begin{frame}{Power Efficiency, Nov 2013}

\begin{center}
\includegraphics[height=2.5in]{Green500evolution}
%"Green500 evolution" by Stefan Parviainen - Own work. Licensed under CC0 via Commons - https://commons.wikimedia.org/wiki/File:Green500_evolution.svg#/media/File:Green500_evolution.svg
\end{center}

\end{frame}


%%%%%%%%%%%%%%%%%%%%%%%%%%%%%%%%%%%%%%%%%%%%%%%%%%%%%%
%%%%%%%%%%%%%%%%%%%%%%%%%%%%%%%%%%%%%%%%%%%%%%%%%%%%%%
\begin{frame}{Where are we going?}
\begin{center}
\includegraphics[height=2.25in]{road}
\end{center}
\end{frame}

%%%%%%%%%%%%%%%%%%%%%%%%%%%%%%%%%%%%%%%%%%%%%%%%%%%%%%
%%%%%%%%%%%%%%%%%%%%%%%%%%%%%%%%%%%%%%%%%%%%%%%%%%%%%%
\section*{}
\begin{frame}{Recap}
\begin{itemize}
\item Humans have been working to use machines for computation for a very long time\vspace*{0.5 em}
\item The 20th century saw the development of the computers we know today $\rightarrow$ development of computational science as a field\vspace*{0.5 em}
\item A revolution in supercomputing began at the end of the 20th century $\rightarrow$ \alert{computational science is a major contributor to knowledge}\vspace*{0.5 em}
\item We are reaching the limits of ``traditional" architecture growth\vspace*{0.5 em}
\item What we can compute and how is tightly tied to computer architecture development
\end{itemize}
\end{frame}

\end{document}