\documentclass[12pt]{article}
\usepackage[top=1in, bottom=1in, left=1in, right=1in]{geometry}

\title{NE 155, Class 3, S14 \\
Types of Equations in the Engineering Fields}
\author{Rachel Slaybaugh}
\date{January 27, 2014}

\usepackage{setspace}
\onehalfspacing

\usepackage{amssymb}
%% The amsthm package provides extended theorem environments
\usepackage{amsthm}
\usepackage{epsfig}
\usepackage{times}
\renewcommand{\ttdefault}{cmtt}
\usepackage{amsmath}
\usepackage{graphicx} % for graphics files

% Draw figures yourself
\usepackage{tikz} 

% The float package HAS to load before hyperref
\usepackage{float} % for psuedocode formatting
\usepackage{xspace}

% from Denovo Methods Manual
\usepackage{mathrsfs}
\usepackage[mathcal]{euscript}
\usepackage{color}
\usepackage{array}

\usepackage[pdftex]{hyperref}

\newcommand{\nth}{n\ensuremath{^{\text{th}}} }
\newcommand{\ve}[1]{\ensuremath{\mathbf{#1}}}


\begin{document}
\begin{center}
{\bf NE 155, Class 3, S14 \\
Types of Equations in the Engineering Fields \\ January 27, 2014}
\end{center}

\setlength{\unitlength}{1in}
\begin{picture}(6,.1) 
\put(0,0) {\line(1,0){6.25}}         
\end{picture}


\noindent \textbf{Introduction}

In science and engineering in general, and nuclear engineering and reactor analysis in specific, we encounter a wide range of mathematical physics equations. In today's lecture we will introduce some of them.

\begin{itemize}
\item Ordinary differential equations (ODEs)
\item Partial differential equations (PDEs)
  \begin{itemize}
  \item Elliptic PDEs
  \item Parabolic PDEs
  \item Hyperbolic PDEs
  \end{itemize}
\item Integro-differential equations
\item Integral equations
\end{itemize}

\section{ODEs}

The most general form of an \nth order linear ordinary differential eqn.\ is
%
\begin{equation}
a_{n+1}(x)y^{(n)} + a_{n}(x)y^{(n-1)} + \cdots + a_{2}(x)y^{(1)} + a_{1}(x)y + a_0 = f(x) \nonumber
\end{equation}
%
\noindent where
\begin{itemize}
\item $a_n$ are coefficients
\item $y^{(n)}$ is the \nth derivative of $y$.
\end{itemize}

Boundary conditions:
\begin{enumerate}
\item if y and its derivatives are given at one end of the domain/interval (time 0 if there's time or x 0 if there's only space, etc.): Initial Value Problem (IVP)
\item if y and/or its derivatives are given at \underline{each} end of the interval: Boundary Value Problem (BVP)
\end{enumerate}

\noindent \textbf{Linear 1st order ODE's}

\noindent \underline{Reminders}
\begin{itemize}
\item Linear means each term is either a constant or the product of a constant and the first power of a single variable (in this case y).
\item the highest derivative is to the first power.
\end{itemize}

%-------------------------------------------------------------
\noindent \underline{1st order ODE Example}:
\begin{equation}
y' + 3y = sin(x) \nonumber
\end{equation}
%
\begin{itemize}
\item IVP if boundary conditions are y(0) = 1; y'(0) = 2
\item BVP if boundary conditions are y(0)=-1, y(1) = 3 on the domain [0,1]
\end{itemize}
%
In this case the general solution is obtained through the use of an integrating factor.

\vspace*{1em}
\noindent \underline{1st order ODE Example}:

Point Kinetics analysis of a nuclear reactor is an IVP, linear 1st order ODE.
%
\begin{align}
\frac{dn(t)}{dt} &= \frac{\rho(t) - \beta}{l^*}n(t) + \sum_{i=1}^{N} \lambda_i C_i(t) \nonumber \\
%
\frac{dC_i(t)}{dt} &= \frac{\beta_i}{l^*}n(t) - \lambda_i C_i(t) \qquad i=1,\dots,N \nonumber
\end{align}
%
Where (we'll talk  more about what these terms mean later)
%
\begin{itemize}
\item n = \# neutrons / s
\item $\beta$ = fraction of delayed neutrons
\item $\lambda_i$ = effective decay constant of the ith precursor
\item $C_i(t)$ = delayed neutron concentration of the ith precursor
\item $l^*$ = mean neutron lifetime
\item $\rho = \frac{k-1}{k}$ = reactivity
\end{itemize}
%
BCs: $n(t=0) = n_0$ and $C_i(t=0) = C_{i,0}$ for $i=1,\dots,N$.

%-------------------------------------------------------------
\vspace*{1em}
\noindent \underline{1st order ODE Example}:

The number of atoms in during radioactive decay (assuming decay only here) is described by the Bateman equation, which is a linear, 1st order ODE that is in a IVP:
%
\begin{align}
\frac{dN_1(t)}{dt} &= -\lambda_1 N_1(t) \nonumber \\
\frac{dN_i(t)}{dt} &= -\lambda_i N_i(t) + \lambda_{i-1}N_{i-1}(t) + \alpha_i F_i \qquad 1 < k < I \nonumber\\
\frac{dN_I(t)}{dt} &= -\lambda_{I-1} N_{I-1}(t) \nonumber \\
\text{BC: }& N_i(t=0) = N_{i,0}\nonumber
\end{align}
%
note: isotope $i$ decays into $i+1$. This can be adapted for decay branches, and becomes more complicated if we have neutrons that transmute isotopes. 

%-------------------------------------------------------------
\vspace*{1em}
\noindent \underline{2nd order ODE Example}:

\begin{align}
-\frac{d}{dx}p(x) \frac{d}{dx}\phi(x) &+ q(x)\phi(x) = S(x) \nonumber \\
\text{defined for }& \alpha \le x \le \beta \nonumber \\
\text{BC: }& \frac{d\phi}{dx} + \gamma \phi = \sigma \qquad \text{at } x=\alpha \text{ and } x = \beta \text{ (BVP)}\nonumber
\end{align}
%
This has
\begin{itemize}
\item Neumann BCs if $\gamma = 0$ (specifies the values that the derivative of a solution is to take on the boundary of the domain)
\item Dirichet BCs: if $\gamma \rightarrow \infty, \sigma / \gamma = $constant then the BD reduces to $\phi = $constant (specifies the values that a solution needs to take on the boundary of the domain)
\item Mixed BCs if $\gamma \ne 0$ (different boundary conditions are used on different parts of the boundary of the domain of the equation.)
\item If $S(x)$ is nonzero at least somewhere over the physical range, a unique solution exists.
\end{itemize}

%-------------------------------------------------------
\vspace*{1 em} \underline{Reminder: eigenpairs}

We can formulate systems of equations as matrix-vector systems that look like \ve{A}$x = \lambda x$. 
\begin{itemize}
\item An eigenvector is a non-zero vector $x$ that, when multiplied by the matrix \ve{A}, yields a constant multiple of $x$. 
\item The constant multiple, $\lambda$, is the eigenvalue corresponding to the eigenvector.
\item There can be more than one eigenvalue, and more than one eigenvector. 
\item Several eigenvectors can have the same eigenvalue
\item Sometime the equation can be reformulated as as $y\ve{A} = \alpha y$ (the left eigenvector). In nuclear, we're used to seeing the right eigenvector formulation.
\end{itemize}
%-------------------------------------------------------------

\vspace*{1em}
\noindent \underline{2nd order ODE Example}:

The homogeneous equation [If  $\phi(x)$  is a solution, so is  $c \phi(x)$, where $c$ is an arbitrary (non-zero) constant. Note that in order for this condition to hold, each term in a linear differential equation of the dependent variable $y$ must contain $y$ or any derivative of $y$; a constant term breaks homogeneity]:
%
\begin{align}
-\frac{d}{dx}p(x) \frac{d}{dx}\phi(x) &+ q(x)\phi(x) = \lambda f(x) \phi(x) \nonumber \\
\text{where }p(x) > 0, f(x) \geq 0, \text{ and }& \lambda = \text{ eigenvalue, and}\nonumber \\
\frac{d\phi}{dx} + \gamma_L \phi &= \sigma \qquad \text{at }x=\alpha \nonumber \\
\frac{d\phi}{dx} + \gamma_R \phi &= \sigma \qquad \text{at }x=\beta \nonumber
\end{align}
%
with $\gamma_L \geq 0$ and $\gamma_R \geq 0$ this is known as the \underline{Sturm-Liouville} eigenvalue problem. If $\gamma_L = \gamma_R = 0$, the BCs become $\phi(\alpha) = \phi(\beta) = 0$.

The solution has an infinite number of eigenfunctions, $\phi_i(x)$ with corresponding \textbf{REAL} and \textbf{DISTINCT} eigenvalues, $\lambda_i$. 

If we number them in sequence: $\lambda_0 < \lambda_1 < \lambda_2 < \dots < \lambda_{\infty}$, then $\lambda_0$ is the lowest eigenvalue and $\phi_0(x)$ is the fundamental eigenmode. 

If $q(x) /geq 0$, then all $\lambda_i$ are positive.

%-------------------------------------------------------------
\vspace*{1em}
\noindent \underline{2nd order ODE Example}:

1-D, 1-group, time-independent neutron diffusion equation:
%
\begin{align}
-\frac{d}{dx}D(x)\frac{d}{dx}\phi(x) + \Sigma_a(x)\phi(x) &= S(x) \qquad \text{Fixed Source} \nonumber \\
-\frac{d}{dx}D(x)\frac{d}{dx}\phi(x) + \Sigma_a(x)\phi(x) &= \frac{1}{k} \nu \Sigma_f(x) \phi(x)\qquad \text{Fission / Eigenvalue} \nonumber
\end{align}
%
BCs: (BVP) vacuum, $phi(\pm a) = 0$

%-------------------------------------------------------------
\section{PDEs}

A partial differential equation is an equation containing an unknown function of two or more variables and its derivatives with respect to those variables. 

\vspace*{1em}
\noindent \underline{Example}:
%
\begin{equation}
\frac{\partial^2 U}{\partial x \partial  y} = 4x + 3y \nonumber
\end{equation}
%
is a 2nd order PDE in two variables. 

If the PDE is linear in U and all derivatives of U, then we say that the PDE is linear.
%
\begin{equation}
A\frac{\partial^2 U}{\partial^2 x} + B\frac{\partial^2 U}{\partial x \partial  y} + C\frac{\partial^2 U}{\partial^2 y} + D\frac{\partial U}{\partial x} + E\frac{\partial U}{\partial y} + FU = G \nonumber
\end{equation}
%
This equation is a 2nd order PDE in two variables. It is linear if $A$ through $G$ depend on $x$ or $y$, but not on $U$.


\end{document}
